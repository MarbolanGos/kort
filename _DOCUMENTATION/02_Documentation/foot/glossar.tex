% Glossar
\newglossaryentry{API} {
	name = API,
	description = {Application Programming Interface, Schnittstelle für die Programmierung}
}

\newglossaryentry{OAuth} {
	name = OAuth,
	description = {OAuth ist ein offenes Protokoll, das eine standardisierte, sichere API-Autorisierung erlaubt\cite{oauth}}
}

\newglossaryentry{CRUD} {
	name = CRUD,
	description = {Das Akronym CRUD beschreibt die 4 Standardoperationen einer Datenbank: \textbf{C}reate, \textbf{R}ead, \textbf{U}pdate, \textbf{D}elete\cite{crud}}
}

\newglossaryentry{AntTarget} {
	name = Ant Target,
	description = {Das Buildautomatisierungs-Tool Ant nennt die einzelnen Schritte eines Builds \emph{Target}. Targets sind eine Sammlung von semantisch zusammengehörigen Tasks. Sie können Abhängigkeiten zu anderen Targets aufweisen, welche dann als Vorbedingung zuerst ausgeführt werden\cite{ant-target}}
}

\newglossaryentry{Cloud} {
	name = Cloud,
	description = {Als Cloud oder Cloud-Computing (\emph{Wolke}) bezeichnet man die Gesamtheit aller Dienste, welche ortsunabhängig im Internet angeboten werden. Dies können zum Beispiel Datenspeicher, Server oder Datenbanken oder schlicht Rechenleistung sein. Der grosse Vorteil der Cloud ist, dass sie sehr leicht skalierbar ist, so dass man  die Leistungen dynamisch an den Bedarf angepassen kann\cite{cloud}}
}

\newglossaryentry{HeadlessBrowser} {
	name = headless Browser,
	description = {Bei einem \emph{headless Browser} handelt es sich um einen Browser, welcher ohne grafische Benutzeroberfläche auskommt. Häufig werden sie dazu verwendet, um Serverjobs, welche auf den Browser angewiesen sind, auszuführen}
}

\newglossaryentry{WebApp} {
	name = Web-App,
	description = {Der Begriff Web-App (von der englischen Kurzform für web application), bezeichnet im allgemeinen Sprachgebrauch Apps für mobile Endgeräte wie Smartphones und Tablet-Computer, die über einen in das Betriebssystem integrierten Browser aus dem Internet geladen und so ohne Installation auf dem mobilen Endgerät genutzt werden können\cite{webapp}}
}

\newglossaryentry{REST} {
	name = REST,
	description = {Representational State Transfer\cite{rest} ist ein Programmierparadigma, welches besagt, dass sich der Zustand einer Webapplikation als Ressource in Form einer URL beschreiben lässt. Auf eine solche Ressourcen können folgende Befehle angewendet werden: \inlinecode{GET}, \inlinecode{POST}, \inlinecode{PUT},\inlinecode{PATCH}, \inlinecode{DELETE}, \inlinecode{HEAD} und \inlinecode{OPTIONS}. 
	HTTP ist ein Protokoll welches REST implementiert}
}

\newglossaryentry{git} {
	name = git,
	description = {Git ist ein verteiltes Versionsverwaltungssystem für Dateien. Ursprünglich wurde es für die Entwicklung des Linux Kernels entwickelt.}
}

\newglossaryentry{ci} {
	name = Continuous Integration,
	description = {Unter dem Begriff \emph{Continuous Integration}\cite{cont-integration} beschreibt die Idee, dass Änderungen an einer Software schnell eingebracht werden sollen. Dazu zählt, dass diese in einem Versionsverwaltungs-Tool eingetragen und durch automatisierte Tests geprüft werden.}
}

\newglossaryentry{Bootstrapping} {
	name = Bootstrapping,
	description = {Bootstrapping wird der Prozess genannt, der auf einem einfachen System ein komplexeres System aktiviert\cite{bootstrapping}. Dadruch wird das System ermächtigt, sich selbst zu starten. Deshalb wird Bootstrapping auch Lösung für das Henne-Ei-Problem genannt.}
}

\newglossaryentry{POI} {
	name = POI,
	description = {Abkürzung für Point of Interest. Dies ist ein allgemeiner Begriff für einen Ort mit irgendeiner Bedeutung, sei es eine Schule, Kirche, Bushaltestelle oder sonst etwas von besonderem Interesse.}
}

\newglossaryentry{Mapper} {
	name = Mapper,
	description = {Personen welche auf OpenStreetMap die Karten ergänzen und pflegen, nennen sich selbst \empf{Mapper}.}
}



