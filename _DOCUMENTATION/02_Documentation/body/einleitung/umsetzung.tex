\chapter{Umsetzung}
\label{umsetzung}

\section{Stand der Technik}
\todo[inline]{Stand der Technik dokumentieren}
Zweck und Inhalt dieses Kapitels.
Was machen andere / welche ähnlichen Arbeiten gibt es zum Thema? Was kann von anderen verwendet werden?

Diese Einleitung soll für den Ingenieur irgendeiner Fachrichtung verständlich sein. Sie stellt die Aufgabe in einen grösseren Zusammenhang und liefert eine genaue Beschreibung der Problemstellung. Allfällige Vorarbeiten oder ähnlich gelagerte Arbeiten werden diskutiert.
 
Theoretische Grundlagen sind nur so weit auszuarbeiten, als dies für die Lösung der Aufgabe nötig ist (keine Lehrbücher schreiben). Die Erkenntnisse aus den theoretischen Untersuchungen sind wenn immer möglich direkt mit der Problemlösung zu verknüpfen.

\section{Vision/Umsetzungskonzept}
\todo[inline]{Vision/Umsetzungskonzept dokumentieren}
Da drin steckt den Kern der Lösungsidee und des Konzeptes

\section{Resultate der Arbeit}
\todo[inline]{Resultate der Arbeit dokumentieren}
Bewertung der Resultate. Vergleich mit anderen Lösungen. 

Was ist Neuartig an der Arbeit? Was ist der Nutzen der Arbeit (quantifizierbar/nicht quantifizierbar)? Was sind die (externen) Kosten der Arbeit?

Zielerreichung: was wurde erreicht, was wurde nicht erreicht bezüglich Kann-/Muss-Zielen? Abweichungen (positiv und negativ) und kurze Begründung dafür.

\section{Schlussfolgerungen und Ausblick}
\todo[inline]{Schlussfolgerungen und Ausblick dokumentieren}
Die Schlussfolgerungen bilden zusammen mit der Zusammenfassung die wichtigsten Abschnitte eines Berichts und sollen daher am sorgfältigsten ausgearbeitet sein. Die Schlussfolgerungen enthalten eine Zusammenfassung und Beurteilung der Resultate (Vergleich mit anderen Lösungen, was wurde erreicht, was nicht, was bleibt noch zu tun, was würde man nun anders tun). 

In den Schlussfolgerungen soll auch ein Ausblick auf das weitere Vorgehen bzw. auf die Bedeutung der erreichten Ergebnisse gegeben werden. 

Weiteres zum Ausblick siehe Kapitel 2.3.9 Weiterentwicklung.