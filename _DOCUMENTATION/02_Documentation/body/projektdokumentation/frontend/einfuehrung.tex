% Einführung
\section{Einführung}

\subsection{Idee}
\brand{OpenStreetMap} ist ein freies Projekt, welches jedermann ermöglicht, Kartendaten zu nutzen und zu editieren.
Durch diesen öffentlichen Charakter ist es nicht ausgeschlossen, dass fehlerhafte bzw. unvollständige Daten eingetragen werden.
So kann es vorkommen, dass neu gezeichnete Objekte (beispielsweise Strassen oder Gebäude) nicht mit allen für sie notwendigen \glspl{Tag} versehen werden.

Diese fehlerhaften Daten zu finden und zu korrigieren kann sehr mühsam und aufwendig sein.

\subsection{Ziel}
Ziel dieser Arbeit war es nun, das Korrigieren von fehlerhaften Daten in \brand{OpenStreetMap} spannend und unterhaltsam zu gestalten.
Die Fehler sollen problemlos zu finden und korrigieren sein.
Dazu werden diese automatisch aufbereitet und auf einer Karte angezeigt werden.

Um die Qualität der eingetragenen Fehlerlösungen sicherzustellen müssen diese erst von weiteren Benutzern validiert werden.
Nach einer definierten Anzahl an positiven Bewertungen sollen die Daten zurück zu \brand{OpenStreetMap} gesendet werden.

Um das Korrigieren der Fehler spannend zu machen, sollen Spielelemente in die App eingebaut werden.
So soll es für die Benutzer möglich sein, Punkte zu sammeln und Auszeichnungen zu gewinnen. Über eine Highscore soll man sich mit anderen Benutzern messen können.