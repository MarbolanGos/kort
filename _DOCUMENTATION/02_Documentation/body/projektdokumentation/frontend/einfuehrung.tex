% Einführung
\section{Einführung}

\subsection{Idee}
Dadurch dass die Daten in \gls{OpenStreetMap} von jedermann bearbeitet werden können, ist es nicht ausgeschlossen, dass diese unvollständig oder gar fehlerhaft eingetragen werden.
So kann es vorkommen, dass neu gezeichnete Objekte (beispielsweise Strassen oder Gebäude) nicht mit allen für sie notwendigen \glspl{Tag} versehen werden.

Dadurch entstehen beispielsweise Strassen ohne Bezeichnung oder mit fehlender Geschwindigkeitslimite.
Diese Daten zu finden und zu korrigieren kann sehr mühsam und aufwendig sein.

\subsection{Ziel}
Ziel dieser Arbeit war es nun, das Korrigieren von fehlerhaften Daten wieder spannend und unterhaltsam zu machen.
Die Fehler sollen einfach zu finden und korrigieren sein.
Dies wird erreicht indem die Fehler automatisch aufbereitet und auf einer Karte angezeigt werden.

Um die Qualität der eingetragenen Daten sicherzustellen müssen Fehlerlösungen erst von weiteren Benutzern validiert werden.
Erst nach einer bestimmten Anzahl an positiven Bewertungen sollen die Daten zurück zu \gls{OpenStreetMap} gesendet werden.

Um das Korrigieren der Fehler spannend zu machen, sollen Spielelemente in die App eingebaut werden.
So soll es für die Benutzer möglich sein Punkte zu sammeln und Auszeichnungen zu gewinnen. Über eine Highscore soll man sich mit anderen Benutzern messen können.