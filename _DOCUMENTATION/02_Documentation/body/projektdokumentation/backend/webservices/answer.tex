% /answer
\subsection{Webservice: Antworten \emph{/answer}}
Bei einigen Fehlertypen wird eine Auswahl an möglichen Antworten vorgegeben.
Um diese Antworten vorzuladen, wird der \emph{Antworten}-Webservice verwendet.
Dieser liefert alle Antworten von allen verschiedenen Fehlertypen zurück.

\subsubsection{Antworten laden}
\begin{table}[H]
\centering
\begin{tabular}{|p{0.15\threecelltabwidth}|p{0.25\threecelltabwidth}|p{0.6\threecelltabwidth}|}
\hline 
\small{\textbf{URL}} & \multicolumn{2}{p{0.85\threecelltabwidth}|}
{
\inlinecode{http://kort.herokuapp.com/server/webservices/answer[/<type>]}
\newline \newline
\inlinecode{<type>} (optional) Antworten auf Typ beschränken
} \\ 
\hline 
\small{\textbf{Methode}} & \multicolumn{2}{p{0.85\threecelltabwidth}|}{\inlinecode{GET}} \\ 
\hline 
\small{\textbf{Parameter}} & \multicolumn{2}{p{0.85\threecelltabwidth}|}{-} \\ 
\hline 
\small{\textbf{Antwort}} & \inlinecode{200 OK} & 
Daten konnten erfolgreich geladen werden. \\
\hline 
\small{\textbf{Antworttyp}} & \multicolumn{2}{p{0.85\threecelltabwidth}|}{\inlinecode{JSON}} \\
\hline 
\end{tabular} 
\caption{Webservice Antworten (GET /answer)}
\end{table}

\textbf{Beispiel:}

\inlinecode{GET http://kort.herokuapp.com/server/webservices/answer/missing\_track\_type}

\textbf{Antwort:}

\lstset{language=JavaScript}
\begin{lstlisting}[style=examples]
{
	"return": [
		{
			"id":"1",
			"value":"grade1",
			"title":"Asphalt, Beton oder Pflastersteine",
			"sorting":"110",
			"type":"missing_track_type"
		},
		{ ... }
	]
}
\end{lstlisting}