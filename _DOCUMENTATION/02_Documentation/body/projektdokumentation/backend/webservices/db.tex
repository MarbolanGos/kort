% /db
\subsection{Webservice: Datenbank \emph{/db}}
\label{webservice-database}
Die Kommunikation vom Webserver zum Datenbankserver geschieht über den \inlinecode{/db}-Webservice.
Als zugehörige Ressource kann eine Tabelle und deren Felder angegeben werden.

Alternativ können eine Reihe von SQL-Queries in einer Transaktion auf der Datenbank laufen gelassen werden.
Dazu gibt es die spezielle \emph{transaction}-Ressource.

Da dieser Webservice direkt die Daten auf der Datenbank verändern kann, ist er durch einen \gls{API}-Key vor fremden Zugriffen geschützt. 
Der \gls{API}-Key muss sowohl auf dem Webserver, wie auch auf dem Datenbankserver in der Umgebungsvariable \inlinecode{KORT\_DB\_API\_KEY} definiert sein.

\subsubsection{SELECT-Abfrage}
\begin{table}[H]
\centering
\begin{tabular}{|p{0.15\threecelltabwidth}|p{0.25\threecelltabwidth}|p{0.6\threecelltabwidth}|}
\hline 
\small{\textbf{URL}} & \multicolumn{2}{p{0.85\threecelltabwidth}|}
{
\inlinecode{http://kort.rdmr.ch/webservices/db/<table>/<fields>}
\newline \newline
\inlinecode{<table>} Datenbank-Tabellenname (inkl. Schema)
\newline
\inlinecode{<fields>} Komma-separierte Liste von Feldern der Tabelle
} \\ 
\hline 
\small{\textbf{Methode}} & \multicolumn{2}{p{0.85\threecelltabwidth}|}{\inlinecode{GET}} \\ 
\hline 
\small{\textbf{Parameter}} & \multicolumn{2}{p{0.85\threecelltabwidth}|}
{
\inlinecode{key} \gls{API}-Key \newline
\inlinecode{where} WHERE-Klausel der Abfrage (Einschränkung) \newline
\inlinecode{orderby} ORDER BY-Klausel der Abfrage (Sortierung) \newline
\inlinecode{limit} Maximale Anzahl Records
} \\ 
\hline 
\small{\textbf{Antwort}} & \inlinecode{200 OK} & 
Die Datenbankabfrage war erfolgreich. \\
\hhline{~--}
 & \inlinecode{403 Forbidden} & 
Der \gls{API}-Key ist nicht korrekt. \\
\hline
\small{\textbf{Antworttyp}} & \multicolumn{2}{p{0.85\threecelltabwidth}|}{\inlinecode{JSON}} \\
\hline 
\end{tabular} 
\caption{Webservice Datenbank (GET /db)}
\end{table}

\textbf{Beispiel:}

\inlinecode{GET http://kort.rdmr.ch/webservices/db/kort.answer/id,value,title?
\\where=type='missing\_track\_type'}

\textbf{Antwort:}

\lstset{language=JavaScript}
\begin{lstlisting}[style=examples]
{
	[
		{
			"id":"1",
			"value":"grade1",
			"title":"Asphalt, Beton oder Pflastersteine"
		},
		{ ... }
	]
}
\end{lstlisting}

\subsubsection{UPDATE-Abfrage}
\begin{table}[H]
\centering
\begin{tabular}{|p{0.15\threecelltabwidth}|p{0.25\threecelltabwidth}|p{0.6\threecelltabwidth}|}
\hline 
\small{\textbf{URL}} & \multicolumn{2}{p{0.85\threecelltabwidth}|}
{
\inlinecode{http://kort.rdmr.ch/webservices/db/<table>/<fields>}
\newline \newline
\inlinecode{<table>} Datenbank-Tabellenname (inkl. Schema)
\newline
\inlinecode{<fields>} Komma-separierte Liste von Feldern der Tabelle
} \\ 
\hline 
\small{\textbf{Methode}} & \multicolumn{2}{p{0.85\threecelltabwidth}|}{\inlinecode{PUT}} \\ 
\hline 
\small{\textbf{Parameter}} & \multicolumn{2}{p{0.85\threecelltabwidth}|}
{
\inlinecode{key} \gls{API}-Key \newline
\inlinecode{where} WHERE-Klausel der Abfrage (Einschränkung) \newline
\inlinecode{return} Komma-separierte Felder welche als Antwort geschicht werden
} \\ 
\hline 
\small{\textbf{Antwort}} & \inlinecode{200 OK} & 
Die Datenbankabfrage war erfolgreich. \\
\hhline{~--}
 & \inlinecode{403 Forbidden} & 
Der \gls{API}-Key ist nicht korrekt. \\
\hline
\small{\textbf{Antworttyp}} & \multicolumn{2}{p{0.85\threecelltabwidth}|}{\inlinecode{JSON}} \\
\hline 
\end{tabular} 
\caption{Webservice Datenbank (PUT /db)}
\end{table}

\textbf{Beispiel:}

\inlinecode{PUT http://kort.rdmr.ch/webservices/db/kort.user/user\_id,username?
\\where=user\_id=3\&return=user\_id,username'}
\begin{lstlisting}[style=examples]
{
	"user_id":3,
	"username":"testuser"
	
}
\end{lstlisting}

\textbf{Antwort:}

\lstset{language=JavaScript}
\begin{lstlisting}[style=examples]
{
	[
		{
			"id":"3",
			"username":"testuser"
		}
	]
}
\end{lstlisting}

\subsubsection{INSERT-Abfrage}
\begin{table}[H]
\centering
\begin{tabular}{|p{0.15\threecelltabwidth}|p{0.25\threecelltabwidth}|p{0.6\threecelltabwidth}|}
\hline 
\small{\textbf{URL}} & \multicolumn{2}{p{0.85\threecelltabwidth}|}
{
\inlinecode{http://kort.rdmr.ch/webservices/db/<table>/<fields>}
\newline \newline
\inlinecode{<table>} Datenbank-Tabellenname (inkl. Schema)
\newline
\inlinecode{<fields>} Komma-separierte Liste von Feldern der Tabelle
} \\ 
\hline 
\small{\textbf{Methode}} & \multicolumn{2}{p{0.85\threecelltabwidth}|}{\inlinecode{POST}} \\ 
\hline 
\small{\textbf{Parameter}} & \multicolumn{2}{p{0.85\threecelltabwidth}|}
{
\inlinecode{key} \gls{API}-Key \newline
\inlinecode{return} Komma-separierte Felder welche als Antwort geschickt werden
} \\ 
\hline 
\small{\textbf{Antwort}} & \inlinecode{200 OK} & 
Die Datenbankabfrage war erfolgreich. \\
\hhline{~--}
 & \inlinecode{403 Forbidden} & 
Der \gls{API}-Key ist nicht korrekt. \\
\hline
\small{\textbf{Antworttyp}} & \multicolumn{2}{p{0.85\threecelltabwidth}|}{\inlinecode{JSON}} \\
\hline 
\end{tabular} 
\caption{Webservice Datenbank (POST /db)}
\end{table}

\textbf{Beispiel:}

\inlinecode{POST http://kort.rdmr.ch/webservices/db/kort.fix/user\_id,error\_id,osm\_id,
\\schema,message?return=fix\_id'}
\begin{lstlisting}[style=examples]
{
	"user_id":5,
	"error_id":8983274,
	"osm_id":212342,
	"schema":"95",
	"message":"grade1"
	
}
\end{lstlisting}

\textbf{Antwort:}

\lstset{language=JavaScript}
\begin{lstlisting}[style=examples]
{
	[
		{
			"fix_id":43
		}
	]
}
\end{lstlisting}

\subsubsection{Transaction}
\begin{table}[H]
\centering
\begin{tabular}{|p{0.15\threecelltabwidth}|p{0.25\threecelltabwidth}|p{0.6\threecelltabwidth}|}
\hline 
\small{\textbf{URL}} & \multicolumn{2}{p{0.85\threecelltabwidth}|}
{
\inlinecode{http://kort.rdmr.ch/webservices/db/transaction}
} \\ 
\hline 
\small{\textbf{Methode}} & \multicolumn{2}{p{0.85\threecelltabwidth}|}{\inlinecode{POST}} \\ 
\hline 
\small{\textbf{Parameter}} & \multicolumn{2}{p{0.85\threecelltabwidth}|}
{
\inlinecode{key} \gls{API}-Key
} \\ 
\hline 
\small{\textbf{Antwort}} & \inlinecode{200 OK} & 
Die Transaktion war erfolgreich. \\
\hhline{~--}
 & \inlinecode{403 Forbidden} & 
Der \gls{API}-Key ist nicht korrekt. \\
\hhline{~--}
 & \inlinecode{404 Not Found} & 
Eine oder mehrere Abfragen der Transaktion waren fehlerhaft. \\
\hline
\small{\textbf{Antworttyp}} & \multicolumn{2}{p{0.85\threecelltabwidth}|}{\inlinecode{JSON}} \\
\hline 
\end{tabular} 
\caption{Webservice Datenbank (POST /db/transaction)}
\end{table}

\textbf{Beispiel:}

\inlinecode{POST http://kort.rdmr.ch/webservices/db/transaction}
\begin{lstlisting}[style=examples]
{
	[
		{
			"type":"SQL",
			"sql":"select vote_koin_count from kort.validations where required_validations > 1"
		},
		{
			"type":"UPDATE",
			"fields":"username",
			"table":"kort.user",
			"where":"user_id=7",
			"returnFields":"user_id,username",
			"data":	{
						"username":"newUsername"
					}
		}
	}
}
\end{lstlisting}

\textbf{Antwort:}

\lstset{language=JavaScript}
\begin{lstlisting}[style=examples]
{
	[
		{
			[
				{
					"vote_koin_count":5
				},
				{ ...}
			]
		},
		{
			"user_id":7,
			"username:"newUsername"
		}
	]
}
\end{lstlisting}