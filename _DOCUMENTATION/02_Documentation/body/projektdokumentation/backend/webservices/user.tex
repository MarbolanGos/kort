% /user
\subsection{Webservice: Benutzer \emph{/user}}
Der Benutzer-Webservice dient zur Authentifizierung des Benutzers.
Über ihn können sich die Benutzer an- und abmelden.
Zudem werden die Benutzerdaten über ihn geladen.

\subsubsection{Benutzerdaten laden}
\begin{table}[H]
\centering
\begin{tabular}{|p{0.15\threecelltabwidth}|p{0.25\threecelltabwidth}|p{0.6\threecelltabwidth}|}
\hline 
\small{\textbf{URL}} & \multicolumn{2}{p{0.85\threecelltabwidth}|}
{
\inlinecode{http://kort.herokuapp.com/server/webservices/user/[<secret>]} 
\newline \newline
\inlinecode{<secret>} (optional) User Secret wird gesendet falls der Benutzer bereits eingeloggt ist.
} \\ 
\hline 
\small{\textbf{Methode}} & \multicolumn{2}{p{0.85\threecelltabwidth}|}{\inlinecode{GET}} \\ 
\hline 
\small{\textbf{Parameter}} & \multicolumn{2}{p{0.85\threecelltabwidth}|}{-} \\ 
\hline 
\small{\textbf{Antwort}} & \inlinecode{200 OK} & 
Daten konnten erfolgreich geladen werden. Der Webservice liefert die Benutzerdaten zurück. \\
\hline 
\small{\textbf{Antworttyp}} & \multicolumn{2}{p{0.85\threecelltabwidth}|}{\inlinecode{JSON}} \\
\hline 
\end{tabular} 
\caption{Webservice Benutzer (GET /user)}
\end{table}

\textbf{Beispiel:}

\inlinecode{GET http://kort.herokuapp.com/server/webservices/user}

\textbf{Antwort:}

\lstset{language=JavaScript}
\begin{lstlisting}[style=examples]
{
	"return": {
		"id":"3",
		"name":"J\u00fcrg Hunziker",
		"username":"tschortsch",
		"oauth_user_id":"email@host.com",
		"oauth_provider":"Google",
		"token":null,
		"fix_count":"2",
		"vote_count":"4",
		"koin_count":"40",
		"secret":"secret",
		"pic_url":"http:\/\/www.gravatar.com\/avatar\/secret?s=200&d=mm&r=r",
		"logged_in":true
	}
}
\end{lstlisting}

\subsubsection{Badges eines Benutzers laden}
\begin{table}[H]
\centering
\begin{tabular}{|p{0.15\threecelltabwidth}|p{0.25\threecelltabwidth}|p{0.6\threecelltabwidth}|}
\hline 
\small{\textbf{URL}} & \multicolumn{2}{p{0.85\threecelltabwidth}|}
{
\inlinecode{http://kort.herokuapp.com/server/webservices/user/<id>/badges} 
\newline \newline
\inlinecode{<id>} ID des Benutzers
} \\ 
\hline 
\small{\textbf{Methode}} & \multicolumn{2}{p{0.85\threecelltabwidth}|}{\inlinecode{GET}} \\ 
\hline 
\small{\textbf{Parameter}} & \multicolumn{2}{p{0.85\threecelltabwidth}|}{-} \\ 
\hline 
\small{\textbf{Antwort}} & \inlinecode{200 OK} & 
Daten konnten erfolgreich geladen werden. Der Webservice liefert alle Badges zurück mit der Angabe, ob der Benutzer ihn gewonnen hat oder nicht. \\
\hline 
\small{\textbf{Antworttyp}} & \multicolumn{2}{p{0.85\threecelltabwidth}|}{\inlinecode{JSON}} \\
\hline 
\end{tabular} 
\caption{Webservice Benutzer (GET /user/<id>/badges)}
\end{table}

\textbf{Beispiel:}

\inlinecode{GET http://kort.herokuapp.com/server/webservices/user/3/badges}

\textbf{Antwort:}

\lstset{language=JavaScript}
\begin{lstlisting}[style=examples]
{
	"return": [
		{
			"id":"1",
			"name":"highscore_place_1",
			"title":"1. Rang",
			"description":"Erster Rang in der Highscore erreicht.",
			"color":"#FFFBCB",
			"sorting":"110",
			"won":true,
			"create_date":"13.12.2012 18:56"
		},
		{ ... }
	]
}
\end{lstlisting}

\subsubsection{Logout}
\begin{table}[H]
\centering
\begin{tabular}{|p{0.15\threecelltabwidth}|p{0.25\threecelltabwidth}|p{0.6\threecelltabwidth}|}
\hline 
\small{\textbf{URL}} & \multicolumn{2}{p{0.85\threecelltabwidth}|}
{
\inlinecode{http://kort.herokuapp.com/server/webservices/user/<id>/logout} 
\newline \newline
\inlinecode{<id>} ID des Benutzers
} \\ 
\hline 
\small{\textbf{Methode}} & \multicolumn{2}{p{0.85\threecelltabwidth}|}{\inlinecode{GET}} \\ 
\hline 
\small{\textbf{Parameter}} & \multicolumn{2}{p{0.85\threecelltabwidth}|}{-} \\ 
\hline 
\small{\textbf{Antwort}} & \inlinecode{200 OK} & 
Der Benutzer wurde erfolgreich ausgeloggt. \\
\hline 
\small{\textbf{Antworttyp}} & \multicolumn{2}{p{0.85\threecelltabwidth}|}{\inlinecode{Text}} \\
\hline 
\end{tabular} 
\caption{Webservice Benutzer (GET /user/<id>/logout)}
\end{table}

\textbf{Beispiel:}

\inlinecode{GET http://kort.herokuapp.com/server/webservices/user/<id>/logout}

\textbf{Antwort:}

\inlinecode{Congratulations! You've now officially logged out!}

\subsubsection{Benutzerdaten ändern}
\begin{table}[H]
\centering
\begin{tabular}{|p{0.15\threecelltabwidth}|p{0.25\threecelltabwidth}|p{0.6\threecelltabwidth}|}
\hline 
\small{\textbf{URL}} & \multicolumn{2}{p{0.85\threecelltabwidth}|}
{
\inlinecode{http://kort.herokuapp.com/server/webservices/user/[<id>]} 
\newline \newline
\inlinecode{<id>} ID des Benutzers
} \\ 
\hline 
\small{\textbf{Methode}} & \multicolumn{2}{p{0.85\threecelltabwidth}|}{\inlinecode{PUT}} \\ 
\hline 
\small{\textbf{Parameter}} & \multicolumn{2}{p{0.85\threecelltabwidth}|}{Die neuen Benutzerdaten müssen als \inlinecode{JSON}-Objekt im Body gesendet werden.} \\ 
\hline 
\small{\textbf{Antwort}} & \inlinecode{200 OK} & 
Der Benutzer wurde erfolgreich aktualisiert. \\
\hline 
\small{\textbf{Antworttyp}} & \multicolumn{2}{p{0.85\threecelltabwidth}|}{\inlinecode{-}} \\
\hline 
\end{tabular} 
\caption{Webservice Benutzer erstellen (PUT /user)}
\end{table}

\textbf{Beispiel:}

\inlinecode{PUT http://kort.herokuapp.com/server/webservices/user/3}
\lstset{language=JavaScript}
\begin{lstlisting}[style=examples]
{
	"logged_in":true,
	"id":"3",
	"username":"tschortsch",
	"oauth_user_id":"user@oauth.com",
	"oauth_provider":"Google",
	"pic_url":"http://www.gravatar.com/avatar/1234?s=200&d=mm&r=r",
	"name":"J\u00fcrg Hunziker",
	"token":null,
	"fix_count":4,
	"vote_count":7,
	"koin_count":85,
	"secret":"secret"
}
\end{lstlisting}

\textbf{Antwort:}
\lstset{language=JavaScript}
\begin{lstlisting}[style=examples]
{
	"user_id":"3",
	"name":"J\u00fcrg Hunziker",
	"username":"tschortsch",
	"oauth_user_id":"user@oauth.com",
	"secret":"secret"
}
\end{lstlisting}