% /validation
\subsection{Webservice: Überprüfung \emph{/validation}}
Die eingetragenen Lösungen werden über den \inlinecode{/validation}-Webservice geladen.
Diesem muss die aktuelle Position mitgeliefert werden, um lediglich die Lösungen in der Umgebung zu laden.

Zusätzlich kann über diesen Webservice eine Überprüfung zu einer Lösung gesendet werden.

\subsubsection{Lösungen laden}
\begin{table}[H]
\centering
\begin{tabular}{|p{0.15\threecelltabwidth}|p{0.25\threecelltabwidth}|p{0.6\threecelltabwidth}|}
\hline 
\small{\textbf{URL}} & \multicolumn{2}{p{0.85\threecelltabwidth}|}
{
\inlinecode{http://kort.herokuapp.com/server/webservices/validation/\newline position/<lat>,<lng>}
\newline \newline
\inlinecode{<lat>} Latitude der aktuellen Position 
\newline
\inlinecode{<lng>} Longitude der aktuellen Position
} \\ 
\hline 
\small{\textbf{Methode}} & \multicolumn{2}{p{0.85\threecelltabwidth}|}{\inlinecode{GET}} \\ 
\hline 
\small{\textbf{Parameter}} & \multicolumn{2}{p{0.85\threecelltabwidth}|}
{
\inlinecode{limit} Maximale Anzahl der zu ladenden Lösungen \newline
\inlinecode{radius} Radius in dem sich die Lösungen befinden müssen
} \\ 
\hline 
\small{\textbf{Antwort}} & \inlinecode{200 OK} & 
Die Lösungen konnten erfolgreich geladen werden. \\
\hline 
\small{\textbf{Antworttyp}} & \multicolumn{2}{p{0.85\threecelltabwidth}|}{\inlinecode{JSON}} \\
\hline 
\end{tabular} 
\caption{Webservice Überprüfung (GET /validation)}
\end{table}

\textbf{Beispiel:}

\inlinecode{GET http://kort.herokuapp.com/server/webservices/validation/position/47.1,8.1?\\limit=1\&radius=5000}

\textbf{Antwort:}

\lstset{language=JavaScript}
\begin{lstlisting}[style=examples]
{
	"return": [
		{
			"id":"186",
			"type":"missing_maxspeed",
			"view_type":"number",
			"fix_user_id":"1",
			"osm_id":"110725957",
			"osm_type":"way",
			"title":"Fehlendes Tempolimit",
			"fixmessage":"50",
			"question":"Darf man auf dieser Strasse mit dieser Geschwindigkeit fahren?",
			"latitude":"47.3370456000000000",
			"longitude":"8.5207856000000000",
			"upratings":"0",
			"downratings":"0",
			"required_validations":"3"
		},
		{ ... }
	]
}
\end{lstlisting}


\subsubsection{Überprüfung eintragen}
\begin{table}[H]
\centering
\begin{tabular}{|p{0.15\threecelltabwidth}|p{0.25\threecelltabwidth}|p{0.6\threecelltabwidth}|}
\hline 
\small{\textbf{URL}} & \multicolumn{2}{p{0.85\threecelltabwidth}|}
{
\inlinecode{http://kort.herokuapp.com/server/webservices/validation/vote}
} \\ 
\hline 
\small{\textbf{Methode}} & \multicolumn{2}{p{0.85\threecelltabwidth}|}{\inlinecode{POST}} \\ 
\hline 
\small{\textbf{Parameter}} & \multicolumn{2}{p{0.85\threecelltabwidth}|}
{Die zu sendende Überprüfung muss als \inlinecode{JSON}-Objekt im Body gesendet werden.} \\ 
\hline 
\small{\textbf{Antwort}} & \inlinecode{200 OK} & 
Die Überprüfung konnte erfolgreich eingetragen werden. Als Antwort werden die erspielten Punkte und Auszeichnungen zurückgeliefert. \\
\hhline{~--} & \inlinecode{403 Forbidden} & 
Der Benutzer ist nicht korrekt eingeloggt und kann somit keine Daten an den Server senden. \\
\hhline{~--} & \inlinecode{400 Bad request} & 
Das gesendete JSON ist nicht valide oder es gab einen Fehler beim Schreiben der Daten in die Datenbank. \\
\hline 
\small{\textbf{Antworttyp}} & \multicolumn{2}{p{0.85\threecelltabwidth}|}{\inlinecode{JSON}} \\
\hline 
\end{tabular} 
\caption{Webservice Überprüfung (POST /validation/vote)}
\end{table}

\subsubsection{Beispiel:}

\inlinecode{POST http://kort.herokuapp.com/server/webservices/validation/vote}
\lstset{language=JavaScript}
\begin{lstlisting}[style=examples]
{
	"id":"ext-record-177",
	"fix_id":151,
	"user_id":"3",
	"valid":"true"
}
\end{lstlisting}

\textbf{Antwort:}

\lstset{language=JavaScript}
\begin{lstlisting}[style=examples]
{
	"badges": [
		{
			"name": "vote_count_10"
		}
	],
	"koin_count_new":"5",
	"koin_count_total":"95"
}
\end{lstlisting}