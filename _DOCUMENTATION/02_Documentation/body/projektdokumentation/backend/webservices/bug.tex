% /bug
\subsection{Webservice: Fehler \emph{/bug}}
Die Fehler werden über den \inlinecode{/bug}-Webservice geladen.
Diesem muss die aktuelle Position mitgeliefert werden.
Dadurch ist es möglich lediglich die Fehler in der Umgebung zu laden.

Zusätzlich kann über diesen Webservice eine Lösung zu einem Fehler eingetragen werden.

\subsubsection{Fehler laden}
\begin{table}[H]
\centering
\begin{tabular}{|p{0.15\threecelltabwidth}|p{0.25\threecelltabwidth}|p{0.6\threecelltabwidth}|}
\hline 
\small{\textbf{URL}} & \multicolumn{2}{p{0.85\threecelltabwidth}|}
{
\inlinecode{http://kort.herokuapp.com/server/webservices/bug/position/ <lat>,<lng>}
\newline \newline
\inlinecode{<lat>} Latitude der aktuellen Position 
\newline
\inlinecode{<lng>} Longitude der aktuellen Position
} \\ 
\hline 
\small{\textbf{Methode}} & \multicolumn{2}{p{0.85\threecelltabwidth}|}{\inlinecode{GET}} \\ 
\hline 
\small{\textbf{Parameter}} & \multicolumn{2}{p{0.85\threecelltabwidth}|}
{
\inlinecode{limit} Maximale Anzahl der zu ladenden Fehler \newline
\inlinecode{radius} Radius in dem sich die Fehler befinden müssen
} \\ 
\hline 
\small{\textbf{Antwort}} & \inlinecode{200 OK} & 
Daten konnten erfolgreich geladen werden. \\
\hline 
\small{\textbf{Antworttyp}} & \multicolumn{2}{p{0.85\threecelltabwidth}|}{\inlinecode{JSON}} \\
\hline 
\end{tabular} 
\caption{Webservice Fehler (GET /bug)}
\end{table}

\textbf{Beispiel:}

\inlinecode{GET http://kort.herokuapp.com/server/webservices/bug/position/47.1,8.1?\\limit=1\&radius=5000}

\textbf{Antwort:}

\lstset{language=JavaScript}
\begin{lstlisting}[style=examples]
{
	"return": [
		{
			"id":"32621371",
			"schema":"95",
			"type":"missing_track_type",
			"osm_id":"119068810",
			"osm_type":"way",
			"title":"Typ des Wegs unbekannt",
			"description":"Um welchen Weg-Typ handelt es sich hier?",
			"latitude":"47.0995850000000000",
			"longitude":"8.0979140000000000",
			"view_type":"select",
			"answer_placeholder":"Typ",
			"fix_koin_count":"5",
			"txt1":"",
			"txt2":"",
			"txt3":"",
			"txt4":"",
			"txt5":""
		},
		{ ... }
	]
}
\end{lstlisting}

\subsubsection{Lösung senden}
\begin{table}[H]
\centering
\begin{tabular}{|p{0.15\threecelltabwidth}|p{0.25\threecelltabwidth}|p{0.6\threecelltabwidth}|}
\hline 
\small{\textbf{URL}} & \multicolumn{2}{p{0.85\threecelltabwidth}|}
{
\inlinecode{http://kort.herokuapp.com/server/webservices/bug/fix}
} \\ 
\hline 
\small{\textbf{Methode}} & \multicolumn{2}{p{0.85\threecelltabwidth}|}{\inlinecode{POST}} \\ 
\hline 
\small{\textbf{Parameter}} & \multicolumn{2}{p{0.85\threecelltabwidth}|}
{Die zu sendende Antwort muss als \inlinecode{JSON}-Objekt im Body gesendet werden.} \\ 
\hline 
\small{\textbf{Antwort}} & \inlinecode{200 OK} & 
Die Lösung konnte erfolgreich gesendet werden. Als Antwort werden die erspielten Punkte und Auszeichnungen zurückgeliefert. \\
\hhline{~--} & \inlinecode{403 Forbidden} & 
Der Benutzer ist nicht korrekt eingeloggt und kann somit keine Daten an den Server senden. \\
\hhline{~--} & \inlinecode{400 Bad request} & 
Das gesendete JSON ist nicht valide oder es gab einen Fehler beim Schreiben der Daten in die Datenbank. \\
\hline 
\small{\textbf{Antworttyp}} & \multicolumn{2}{p{0.85\threecelltabwidth}|}{\inlinecode{JSON}} \\
\hline 
\end{tabular} 
\caption{Webservice Fehler (POST /bug)}
\end{table}

\textbf{Beispiel:}

\inlinecode{POST http://kort.herokuapp.com/server/webservices/bug/fix}
\lstset{language=JavaScript}
\begin{lstlisting}[style=examples]
{
	"id":"ext-record-230",
	"user_id":3,
	"error_id":"28704192",
	"schema":"95",
	"osm_id":1611867263,
	"message":"McDonalds"
}
\end{lstlisting}

\textbf{Antwort:}

\lstset{language=JavaScript}
\begin{lstlisting}[style=examples]
{
	"badges": [
		{
			"name": "highscore_place_1"
		}
	],
	"koin_count_new":"15",
	"koin_count_total":"55"
}
\end{lstlisting}