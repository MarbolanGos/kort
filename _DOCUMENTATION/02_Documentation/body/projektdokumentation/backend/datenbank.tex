\section{Datenbank}
Die Datenbank ist in verschiedenen Schemas organisiert.
Das Schema \inlinecode{kort} beinhaltet die Daten und Views für die Applikation.

\subsection{Views}
Um die Applikationslogik möglichst von der Datenbank unabhängig zu halten, wird aus dem Backend-Code heraus nur auf die Views aus dem Schema \inlinecode{kort} zugegriffen.
Dies ermöglicht es weitere Schemas unabhängig voneinander hinzuzufügen und wieder zu entfernen.
Auch Änderungen innerhalb des Schemas bedeuten keinen Bruch des View-\glspl{API}.

\subsection{Applikationsschema kort}
Als Grundregel verwenden wir immer Views als \gls{API} gegen das wir programmieren.
So lassen sich Änderungen daran sehr leicht einbringen einbringen, ohne dafür eine Tabelle anzupassen.

Jedes Model (siehe Abschnitt \ref{kort-store-model-package})  in der \gls{WebApp} hat seine eigene View in der Datenbank, welche genau die Daten liefert, welche es braucht.
Dies reduziert den Code für das Frontend und verschiebt die Datenlogik in die Datenbank.

\subsection{Schema für Fehlerquellen}
Jede Fehlerquelle sollte ein eigenes Schema besitzen.
Jegliche Fehlerquellen-spezifischen Daten, Funktionen oder Types sind dann in diesem Schema.
Dadurch lassen sich diese beliebig hinzufügen und entfernen.

In unserem Fall kommt uns das zugute, da jede Nacht das \emph{KeepRight} Schema gelöscht wird und anschliessend mit den neuen Fehlerdaten neu angelegt wird.

So ist sichergestellt, dass alle zugehörigen Informationen beieinander sind.
Lediglich in der View \inlinecode{kort.all\_errors} werden die Fehlerdaten der verschiedenen Schemas zusammengezogen.
Die Schemas für Fehlerquellen gehören somit nicht zum \gls{API} der Datenbank.

\subsection{Transaktionen}
Um die Konsistenz und Integrität der Daten zu wahren, war es eine wichtige Anforderung Transaktionen auf der Datenbank durchführen zu können.
So sollten beispielsweise einem Benutzer \emph{Koins} gutgeschrieben werden, wenn er einen Lösungsvorschlag abliefert.

Dadurch, dass die Datenbank über eine \gls{REST}-Schnittstelle angesprochen wird, könnte man auf die Idee kommen, mehrere Datenbankabfragen mit nacheinander abgeschickten Requests zu realisieren.
Dadurch könnte die Datenbank aber zwischen den einzelnen Requests in einen nicht-definierten Zustand geraten.

Um dies zu verhindern ist es notwendig Transaktionen über eine separate \gls{REST}-Ressource zu abstrahieren.
Clients können dazu ihre Abfragen in einem einzigen Request schicken. Diese werden dann in einer einzigen Datenbank-Transaktion ausgeführt (siehe Abschnitt \ref{webservice-database}).
Falls eine Abfrage schief läuft, werden automatisch die Änderungen der Transaktion rückgängig gemacht (\emph{Alles oder nichts}).