\section{REST-Schnittellen}
Die Entscheidung \gls{REST}-Schnittstellen zu verwenden haben wir schnell gefasst. 
Zum einen ist damit eine einheitliche Schnittstelle im gesamten System vorhanden, so dass immer klar ist über welchen Kanal eine Kommunikation passiert.
Zum anderen sind \gls{REST}-Schnittellen für Webapplikation sehr einfach zu verwenden, da entsprechende Bibliotheksfunktionen bereits vorhanden sind.
Schlussendlich ist entscheidend, dass \gls{REST}-Schnittstellen ein grosses Mass an Plattformunabhängigkeit bieten und so die räumliche Teilung der Systeme erleichtert.

\subsection{Slim Framework}
\todo[inline]{Slim}

% Webservice /answer
% /answer
\subsection{Webservice: Antworten \emph{/answer}}
Bei einigen Fehlertypen wird eine Auswahl an möglichen Antworten vorgegeben.
Um diese Antworten vorzuladen, wird der \emph{Antworten}-Webservice verwendet.
Dieser liefert alle Antworten von allen verschiedenen Fehlertypen zurück.

\subsubsection{Antworten laden}
\begin{table}[H]
\centering
\begin{tabular}{|p{0.15\threecelltabwidth}|p{0.25\threecelltabwidth}|p{0.6\threecelltabwidth}|}
\hline 
\small{\textbf{URL}} & \multicolumn{2}{p{0.85\threecelltabwidth}|}
{
\inlinecode{http://kort.herokuapp.com/server/webservices/answer[/<type>]}
\newline \newline
\inlinecode{<type>} (optional) Antworten auf Typ beschränken
} \\ 
\hline 
\small{\textbf{Methode}} & \multicolumn{2}{p{0.85\threecelltabwidth}|}{\inlinecode{GET}} \\ 
\hline 
\small{\textbf{Parameter}} & \multicolumn{2}{p{0.85\threecelltabwidth}|}{-} \\ 
\hline 
\small{\textbf{Antwort}} & \inlinecode{200 OK} & 
Daten konnten erfolgreich geladen werden. \\
\hline 
\small{\textbf{Antworttyp}} & \multicolumn{2}{p{0.85\threecelltabwidth}|}{\inlinecode{JSON}} \\
\hline 
\end{tabular} 
\caption{Webservice Antworten (GET /answer)}
\end{table}

\textbf{Beispiel:}

\inlinecode{GET http://kort.herokuapp.com/server/webservices/answer/missing\_track\_type}

\textbf{Antwort:}

\lstset{language=JavaScript}
\begin{lstlisting}[style=examples]
{
	"return": [
		{
			"id":"1",
			"value":"grade1",
			"title":"Asphalt, Beton oder Pflastersteine",
			"sorting":"110",
			"type":"missing_track_type"
		},
		{ ... }
	]
}
\end{lstlisting}

% Webservice /bug
\subsection{Webservice: Fehler \emph{/bug}}
Die Fehler werden über den \inlinecode{/bug}-Webservice geladen.
Diesem muss die aktuelle Position mitgeliefert werden.
Dadurch ist es möglich, lediglich die Fehler in der Umgebung zu laden.

Zusätzlich kann über diesen Webservice eine Lösung zu einem Fehler eingetragen werden.

\subsubsection{Fehler laden}
\begin{table}[H]
\centering
\begin{tabular}{|p{0.15\threecelltabwidth}|p{0.25\threecelltabwidth}|p{0.6\threecelltabwidth}|}
\hline 
\small{\textbf{URL}} & \multicolumn{2}{p{0.85\threecelltabwidth}|}
{
\inlinecode{http://kort.herokuapp.com/server/webservices/bug/position/ <lat>,<lng>}
\newline \newline
\inlinecode{<lat>} Latitude der aktuellen Position 
\newline
\inlinecode{<lng>} Longitude der aktuellen Position
} \\ 
\hline 
\small{\textbf{Methode}} & \multicolumn{2}{p{0.85\threecelltabwidth}|}{\inlinecode{GET}} \\ 
\hline 
\small{\textbf{Parameter}} & \multicolumn{2}{p{0.85\threecelltabwidth}|}
{
\inlinecode{limit} Maximale Anzahl der zu ladenden Fehler \newline
\inlinecode{radius} Radius in dem sich die Fehler befinden müssen
} \\ 
\hline 
\small{\textbf{Antwort}} & \inlinecode{200 OK} & 
Daten konnten erfolgreich geladen werden. \\
\hline 
\small{\textbf{Antworttyp}} & \multicolumn{2}{p{0.85\threecelltabwidth}|}{\inlinecode{JSON}} \\
\hline 
\end{tabular} 
\caption{Webservice Fehler (GET /bug)}
\end{table}

\textbf{Beispiel:}

\inlinecode{GET http://kort.herokuapp.com/server/webservices/bug/position/47.1,8.1?\\limit=1\&radius=5000}

\textbf{Antwort:}

\lstset{language=JavaScript}
\begin{lstlisting}[style=examples]
{
	"return": [
		{
			"id":"32621371",
			"schema":"95",
			"type":"missing_track_type",
			"osm_id":"119068810",
			"osm_type":"way",
			"title":"Typ des Wegs unbekannt",
			"description":"Um welchen Weg-Typ handelt es sich hier?",
			"latitude":"47.0995850000000000",
			"longitude":"8.0979140000000000",
			"view_type":"select",
			"answer_placeholder":"Typ",
			"fix_koin_count":"5",
			"txt1":"",
			"txt2":"",
			"txt3":"",
			"txt4":"",
			"txt5":""
		},
		{ ... }
	]
}
\end{lstlisting}

\subsubsection{Lösung senden}
\begin{table}[H]
\centering
\begin{tabular}{|p{0.15\threecelltabwidth}|p{0.25\threecelltabwidth}|p{0.6\threecelltabwidth}|}
\hline 
\small{\textbf{URL}} & \multicolumn{2}{p{0.85\threecelltabwidth}|}
{
\inlinecode{http://kort.herokuapp.com/server/webservices/bug/fix}
} \\ 
\hline 
\small{\textbf{Methode}} & \multicolumn{2}{p{0.85\threecelltabwidth}|}{\inlinecode{POST}} \\ 
\hline 
\small{\textbf{Parameter}} & \multicolumn{2}{p{0.85\threecelltabwidth}|}
{Die zu sendende Antwort muss als \inlinecode{JSON}-Objekt im Body gesendet werden.} \\ 
\hline 
\small{\textbf{Antwort}} & \inlinecode{200 OK} & 
Die Lösung konnte erfolgreich gesendet werden. Als Antwort werden die erspielten Punkte und Auszeichnungen zurückgeliefert. \\
\hhline{~--} & \inlinecode{403 Forbidden} & 
Der Benutzer ist nicht korrekt eingeloggt und kann somit keine Daten an den Server senden. \\
\hhline{~--} & \inlinecode{400 Bad request} & 
Das gesendete \inlinecode{JSON} ist nicht valide oder es gab einen Fehler beim Schreiben der Daten in die Datenbank. \\
\hline 
\small{\textbf{Antworttyp}} & \multicolumn{2}{p{0.85\threecelltabwidth}|}{\inlinecode{JSON}} \\
\hline 
\end{tabular} 
\caption{Webservice Fehler (POST /bug)}
\end{table}

\textbf{Beispiel:}

\inlinecode{POST http://kort.herokuapp.com/server/webservices/bug/fix}
\lstset{language=JavaScript}
\begin{lstlisting}[style=examples]
{
	"id":"ext-record-230",
	"user_id":3,
	"error_id":"28704192",
	"schema":"95",
	"osm_id":1611867263,
	"message":"McDonalds"
}
\end{lstlisting}

\textbf{Antwort:}

\lstset{language=JavaScript}
\begin{lstlisting}[style=examples]
{
	"badges": [
		{
			"name": "highscore_place_1"
		}
	],
	"koin_count_new":"15",
	"koin_count_total":"55"
}
\end{lstlisting}

% Webservice /highscore
% /highscore
\subsection{Webservice: Highscore \emph{/highscore}}
Über den Highscore-Webservice können die Benutzer sortiert nach Anzahl \emph{koins} geladen werden.

\subsubsection{Highscore laden}
\begin{table}[H]
\centering
\begin{tabular}{|p{0.15\threecelltabwidth}|p{0.25\threecelltabwidth}|p{0.6\threecelltabwidth}|}
\hline 
\small{\textbf{URL}} & \multicolumn{2}{p{0.85\threecelltabwidth}|}
{
\inlinecode{http://kort.herokuapp.com/server/webservices/highscore}
} \\ 
\hline 
\small{\textbf{Methode}} & \multicolumn{2}{p{0.85\threecelltabwidth}|}{\inlinecode{GET}} \\ 
\hline 
\small{\textbf{Parameter}} & \multicolumn{2}{p{0.85\threecelltabwidth}|}{\inlinecode{limit} Maximale Anzahl der Benutzer} \\ 
\hline 
\small{\textbf{Antwort}} & \inlinecode{200 OK} & 
Daten konnten erfolgreich geladen werden. \\
\hline 
\small{\textbf{Antworttyp}} & \multicolumn{2}{p{0.85\threecelltabwidth}|}{\inlinecode{JSON}} \\
\hline 
\end{tabular} 
\caption{Webservice Antworten (/highscore)}
\end{table}

\textbf{Beispiel:}

\inlinecode{GET http://kort.herokuapp.com/server/webservices/highscore?limit=10}

\textbf{Antwort:}

\lstset{language=JavaScript}
\begin{lstlisting}[style=examples]
{
	"return": [
		{
			"user_id":"3",
			"username":"tschortsch",
			"koin_count":"140",
			"fix_count":"12",
			"vote_count":"4",
			"ranking":"1",
			"you":true
		},
		{ ... }
	]
}
\end{lstlisting}

% Webservice /osm
% /osm
\subsection{Webservice: OpenStreetMap \emph{/osm}}
Um \gls{OpenStreetMap}-Objekte auf der Karte anzuzeigen, werden über den \inlinecode{/osm}-Webservice die entsprechenden OSM-Daten geladen.
Der Webservice leitet den Request an das OSM API\footnote{\url{http://wiki.openstreetmap.org/wiki/API_v0.6}} weiter und sendet das Resultat an die Webapplikation zurück.

\subsubsection{OpenStreetMap Objekt laden}
\begin{table}[H]
\centering
\begin{tabular}{|p{0.15\threecelltabwidth}|p{0.25\threecelltabwidth}|p{0.6\threecelltabwidth}|}
\hline 
\small{\textbf{URL}} & \multicolumn{2}{p{0.85\threecelltabwidth}|}
{
\inlinecode{http://kort.herokuapp.com/server/webservices/osm/<type>/<id>} 
\newline \newline
\inlinecode{<type>} OSM-Objekttyp
\newline
\inlinecode{<id>} ID des OSM-Objekts
} \\ 
\hline 
\small{\textbf{Methode}} & \multicolumn{2}{p{0.85\threecelltabwidth}|}{\inlinecode{GET}} \\ 
\hline 
\small{\textbf{Parameter}} & \multicolumn{2}{p{0.85\threecelltabwidth}|}{-} \\ 
\hline 
\small{\textbf{Antwort}} & \inlinecode{200 OK} & 
Daten konnten erfolgreich geladen werden. \\
\hline 
\small{\textbf{Antworttyp}} & \multicolumn{2}{p{0.85\threecelltabwidth}|}{\inlinecode{XML}} \\
\hline 
\end{tabular} 
\caption{Webservice OpenStreetMap (/osm)}
\end{table}

\textbf{Beispiel:}

\inlinecode{GET http://kort.herokuapp.com/server/webservices/osm/node/1658024260}

\textbf{Antwort:}

\lstset{language=XML}
\begin{lstlisting}[style=examples]
<?xml version="1.0" encoding="UTF-8"?>
<osm version="0.6" generator="OpenStreetMap server" copyright="OpenStreetMap and contributors" attribution="http://www.openstreetmap.org/copyright" license="http://opendatacommons.org/licenses/odbl/1-0/">
  <node id="1658024260" version="1" changeset="10861664" lat="47.5114378" lon="8.5443127" user="pfrauenf" uid="479871" visible="true" timestamp="2012-03-03T20:05:48Z">
    <tag k="amenity" v="fast_food"/>
  </node>
</osm>
\end{lstlisting}

% Webservice /user
% /user
\subsection{Webservice: Benutzer \emph{/user}}
Der Benutzer-Webservice dient zur Authentifizierung des Benutzers.
Über ihn können sich die Benutzer an- und abmelden.
Zudem werden die Benutzerdaten über ihn geladen.

\subsubsection{Benutzerdaten laden}
\begin{table}[H]
\centering
\begin{tabular}{|p{0.15\threecelltabwidth}|p{0.25\threecelltabwidth}|p{0.6\threecelltabwidth}|}
\hline 
\small{\textbf{URL}} & \multicolumn{2}{p{0.85\threecelltabwidth}|}
{
\inlinecode{http://kort.herokuapp.com/server/webservices/user/[<secret>]} 
\newline \newline
\inlinecode{<secret>} (optional) User Secret wird gesendet falls der Benutzer bereits eingeloggt ist.
} \\ 
\hline 
\small{\textbf{Methode}} & \multicolumn{2}{p{0.85\threecelltabwidth}|}{\inlinecode{GET}} \\ 
\hline 
\small{\textbf{Parameter}} & \multicolumn{2}{p{0.85\threecelltabwidth}|}{-} \\ 
\hline 
\small{\textbf{Antwort}} & \inlinecode{200 OK} & 
Daten konnten erfolgreich geladen werden. Der Webservice liefert die Benutzerdaten zurück. \\
\hline 
\small{\textbf{Antworttyp}} & \multicolumn{2}{p{0.85\threecelltabwidth}|}{\inlinecode{JSON}} \\
\hline 
\end{tabular} 
\caption{Webservice Benutzer (/user)}
\end{table}

\textbf{Beispiel:}

\inlinecode{GET http://kort.herokuapp.com/server/webservices/user}

\textbf{Antwort:}

\lstset{language=JavaScript}
\begin{lstlisting}[style=examples]
{
	"return": {
		"id":"3",
		"name":"J\u00fcrg Hunziker",
		"username":"tschortsch",
		"oauth_user_id":"email@host.com",
		"oauth_provider":"Google",
		"token":null,
		"fix_count":"2",
		"vote_count":"4",
		"koin_count":"40",
		"secret":"secret",
		"pic_url":"http:\/\/www.gravatar.com\/avatar\/secret?s=200&d=mm&r=r",
		"logged_in":true
	}
}
\end{lstlisting}

\subsubsection{Badges eines Benutzers laden}
\begin{table}[H]
\centering
\begin{tabular}{|p{0.15\threecelltabwidth}|p{0.25\threecelltabwidth}|p{0.6\threecelltabwidth}|}
\hline 
\small{\textbf{URL}} & \multicolumn{2}{p{0.85\threecelltabwidth}|}
{
\inlinecode{http://kort.herokuapp.com/server/webservices/user/<id>/badges} 
\newline \newline
\inlinecode{<id>} ID des Benutzers
} \\ 
\hline 
\small{\textbf{Methode}} & \multicolumn{2}{p{0.85\threecelltabwidth}|}{\inlinecode{GET}} \\ 
\hline 
\small{\textbf{Parameter}} & \multicolumn{2}{p{0.85\threecelltabwidth}|}{-} \\ 
\hline 
\small{\textbf{Antwort}} & \inlinecode{200 OK} & 
Daten konnten erfolgreich geladen werden. Der Webservice liefert alle Badges zurück mit der Angabe, ob der Benutzer ihn gewonnen hat oder nicht. \\
\hline 
\small{\textbf{Antworttyp}} & \multicolumn{2}{p{0.85\threecelltabwidth}|}{\inlinecode{JSON}} \\
\hline 
\end{tabular} 
\caption{Webservice Benutzer (/user/<id>/badges)}
\end{table}

\textbf{Beispiel:}

\inlinecode{GET http://kort.herokuapp.com/server/webservices/user/3/badges}

\textbf{Antwort:}

\lstset{language=JavaScript}
\begin{lstlisting}[style=examples]
{
	"return": [
		{
			"id":"1",
			"name":"highscore_place_1",
			"title":"1. Rang",
			"description":"Erster Rang in der Highscore erreicht.",
			"color":"#FFFBCB",
			"sorting":"110",
			"won":true,
			"create_date":"13.12.2012 18:56"
		},
		{ ... }
	]
}
\end{lstlisting}

\subsubsection{Logout}
\begin{table}[H]
\centering
\begin{tabular}{|p{0.15\threecelltabwidth}|p{0.25\threecelltabwidth}|p{0.6\threecelltabwidth}|}
\hline 
\small{\textbf{URL}} & \multicolumn{2}{p{0.85\threecelltabwidth}|}
{
\inlinecode{http://kort.herokuapp.com/server/webservices/user/<id>/logout} 
\newline \newline
\inlinecode{<id>} ID des Benutzers
} \\ 
\hline 
\small{\textbf{Methode}} & \multicolumn{2}{p{0.85\threecelltabwidth}|}{\inlinecode{GET}} \\ 
\hline 
\small{\textbf{Parameter}} & \multicolumn{2}{p{0.85\threecelltabwidth}|}{-} \\ 
\hline 
\small{\textbf{Antwort}} & \inlinecode{200 OK} & 
Der Benutzer wurde erfolgreich ausgeloggt. \\
\hline 
\small{\textbf{Antworttyp}} & \multicolumn{2}{p{0.85\threecelltabwidth}|}{\inlinecode{Text}} \\
\hline 
\end{tabular} 
\caption{Webservice Benutzer (/user/<id>/logout)}
\end{table}

\textbf{Beispiel:}

\inlinecode{GET http://kort.herokuapp.com/server/webservices/user/<id>/logout}

\textbf{Antwort:}

\inlinecode{Congratulations! You've now officially logged out!}

\subsubsection{Benutzerdaten ändern}
\begin{table}[H]
\centering
\begin{tabular}{|p{0.15\threecelltabwidth}|p{0.25\threecelltabwidth}|p{0.6\threecelltabwidth}|}
\hline 
\small{\textbf{URL}} & \multicolumn{2}{p{0.85\threecelltabwidth}|}
{
\inlinecode{http://kort.herokuapp.com/server/webservices/user/[<id>]} 
\newline \newline
\inlinecode{<id>} ID des Benutzers
} \\ 
\hline 
\small{\textbf{Methode}} & \multicolumn{2}{p{0.85\threecelltabwidth}|}{\inlinecode{PUT}} \\ 
\hline 
\small{\textbf{Parameter}} & \multicolumn{2}{p{0.85\threecelltabwidth}|}{Die neuen Benutzerdaten müssen als \inlinecode{JSON}-Objekt im Body gesendet werden.} \\ 
\hline 
\small{\textbf{Antwort}} & \inlinecode{200 OK} & 
Der Benutzer wurde erfolgreich aktualisiert. \\
\hline 
\small{\textbf{Antworttyp}} & \multicolumn{2}{p{0.85\threecelltabwidth}|}{\inlinecode{-}} \\
\hline 
\end{tabular} 
\caption{Webservice Benutzer erstellen (/user)}
\end{table}

\textbf{Beispiel:}

\inlinecode{PUT http://kort.herokuapp.com/server/webservices/user/3}
\lstset{language=JavaScript}
\begin{lstlisting}[style=examples]
{
	"logged_in":true,
	"id":"3",
	"username":"tschortsch",
	"oauth_user_id":"user@oauth.com",
	"oauth_provider":"Google",
	"pic_url":"http://www.gravatar.com/avatar/1234?s=200&d=mm&r=r",
	"name":"J\u00fcrg Hunziker",
	"token":null,
	"fix_count":4,
	"vote_count":7,
	"koin_count":85,
	"secret":"secret"
}
\end{lstlisting}

\textbf{Antwort:}
\lstset{language=JavaScript}
\begin{lstlisting}[style=examples]
{
	"user_id":"3",
	"name":"J\u00fcrg Hunziker",
	"username":"tschortsch",
	"oauth_user_id":"user@oauth.com",
	"secret":"secret"
}
\end{lstlisting}

% Webservice /user
% /validation
\subsection{Webservice: Überprüfung \emph{/validation}}
Die eingetragenen Lösungen werden über den \inlinecode{/validation}-Webservice geladen.
Diesem muss die aktuelle Position mitgeliefert werden, um lediglich die Lösungen in der Umgebung zu laden.

Zusätzlich kann über diesen Webservice eine Überprüfung zu einer Lösung gesendet werden.

\subsubsection{Lösungen laden}
\begin{table}[H]
\centering
\begin{tabular}{|p{0.15\threecelltabwidth}|p{0.25\threecelltabwidth}|p{0.6\threecelltabwidth}|}
\hline 
\small{\textbf{URL}} & \multicolumn{2}{p{0.85\threecelltabwidth}|}
{
\inlinecode{http://kort.herokuapp.com/server/webservices/validation/\newline position/<lat>,<lng>}
\newline \newline
\inlinecode{<lat>} Latitude der aktuellen Position 
\newline
\inlinecode{<lng>} Longitude der aktuellen Position
} \\ 
\hline 
\small{\textbf{Methode}} & \multicolumn{2}{p{0.85\threecelltabwidth}|}{\inlinecode{GET}} \\ 
\hline 
\small{\textbf{Parameter}} & \multicolumn{2}{p{0.85\threecelltabwidth}|}
{
\inlinecode{limit} Maximale Anzahl der zu ladenden Lösungen \newline
\inlinecode{radius} Radius, in dem sich die Lösungen befinden müssen
} \\ 
\hline 
\small{\textbf{Antwort}} & \inlinecode{200 OK} & 
Die Lösungen konnten erfolgreich geladen werden. \\
\hline 
\small{\textbf{Antworttyp}} & \multicolumn{2}{p{0.85\threecelltabwidth}|}{\inlinecode{JSON}} \\
\hline 
\end{tabular} 
\caption{Webservice Überprüfung (GET /validation)}
\end{table}

\textbf{Beispiel:}

\inlinecode{GET http://kort.herokuapp.com/server/webservices/validation/position/47.1,8.1?\\limit=1\&radius=5000}

\textbf{Antwort:}

\lstset{language=JavaScript}
\begin{lstlisting}[style=examples]
{
	"return": [
		{
			"id":"186",
			"type":"missing_maxspeed",
			"view_type":"number",
			"fix_user_id":"1",
			"osm_id":"110725957",
			"osm_type":"way",
			"title":"Fehlendes Tempolimit",
			"fixmessage":"50",
			"question":"Darf man auf dieser Strasse mit dieser Geschwindigkeit fahren?",
			"latitude":"47.3370456000000000",
			"longitude":"8.5207856000000000",
			"upratings":"0",
			"downratings":"0",
			"required_validations":"3"
		},
		{ ... }
	]
}
\end{lstlisting}


\subsubsection{Überprüfung eintragen}
\begin{table}[H]
\centering
\begin{tabular}{|p{0.15\threecelltabwidth}|p{0.25\threecelltabwidth}|p{0.6\threecelltabwidth}|}
\hline 
\small{\textbf{URL}} & \multicolumn{2}{p{0.85\threecelltabwidth}|}
{
\inlinecode{http://kort.herokuapp.com/server/webservices/validation/vote}
} \\ 
\hline 
\small{\textbf{Methode}} & \multicolumn{2}{p{0.85\threecelltabwidth}|}{\inlinecode{POST}} \\ 
\hline 
\small{\textbf{Parameter}} & \multicolumn{2}{p{0.85\threecelltabwidth}|}
{Die zu sendende Überprüfung muss als \inlinecode{JSON}-Objekt im Body gesendet werden.} \\ 
\hline 
\small{\textbf{Antwort}} & \inlinecode{200 OK} & 
Die Überprüfung konnte erfolgreich eingetragen werden. Als Antwort werden die erspielten Punkte und Auszeichnungen zurückgeliefert. \\
\hhline{~--} & \inlinecode{403 Forbidden} & 
Der Benutzer ist nicht korrekt eingeloggt und kann somit keine Daten an den Server senden. \\
\hhline{~--} & \inlinecode{400 Bad request} & 
Das gesendete JSON ist nicht valide oder es gab einen Fehler beim Schreiben der Daten in die Datenbank. \\
\hline 
\small{\textbf{Antworttyp}} & \multicolumn{2}{p{0.85\threecelltabwidth}|}{\inlinecode{JSON}} \\
\hline 
\end{tabular} 
\caption{Webservice Überprüfung (POST /validation/vote)}
\end{table}

\subsubsection{Beispiel:}

\inlinecode{POST http://kort.herokuapp.com/server/webservices/validation/vote}
\lstset{language=JavaScript}
\begin{lstlisting}[style=examples]
{
	"id":"ext-record-177",
	"fix_id":151,
	"user_id":"3",
	"valid":"true"
}
\end{lstlisting}

\textbf{Antwort:}

\lstset{language=JavaScript}
\begin{lstlisting}[style=examples]
{
	"badges": [
		{
			"name": "vote_count_10"
		}
	],
	"koin_count_new":"5",
	"koin_count_total":"95"
}
\end{lstlisting}


\todo[inline]{/db Webservice dokumentieren}