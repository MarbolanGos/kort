\section{REST-Schnittellen}
Die Entscheidung \gls{REST}-Schnittstellen zu verwenden haben wir schnell gefasst. 
Zum einen ist damit eine einheitliche Schnittstelle im gesamten System vorhanden, so dass immer klar ist über welchen Kanal eine Kommunikation passiert.
Zum anderen sind \gls{REST}-Schnittellen für Webapplikation sehr einfach zu verwenden, da entsprechende Bibliotheksfunktionen bereits vorhanden sind.
Schlussendlich ist entscheidend, dass \gls{REST}-Schnittstellen ein grosses Mass an Plattformunabhängigkeit bieten und so die räumliche Teilung der Systeme erleichtert.

\subsection{Slim Framework}
\todo[inline]{Slim}

% /answer
\subsection{Webservice: Antworten \emph{/answer}}
Bei einigen Fehlertypen wird eine Auswahl an möglichen Antworten vorgegeben.
Um diese Antworten vorzuladen, wird der \emph{Antworten}-Webservice verwendet.
Dieser liefert alle Antworten von allen verschiedenen Fehlertypen zurück.

\begin{table}[H]
\centering
\begin{tabular}{|p{0.15\threecelltabwidth}|p{0.25\threecelltabwidth}|p{0.6\threecelltabwidth}|}
\hline 
\small{\textbf{URL}} & \multicolumn{2}{p{0.85\threecelltabwidth}|}
{
\inlinecode{http://kort.herokuapp.com/server/webservices/answer[/<type>]}
\newline \newline
\inlinecode{<type>} (optional) Antworten auf Typ beschränken
} \\ 
\hline 
\small{\textbf{Methode}} & \multicolumn{2}{p{0.85\threecelltabwidth}|}{\inlinecode{GET}} \\ 
\hline 
\small{\textbf{Parameter}} & \multicolumn{2}{p{0.85\threecelltabwidth}|}{-} \\ 
\hline 
\small{\textbf{Antwort}} & \inlinecode{200 OK} & 
Daten konnten erfolgreich geladen werden. \\
\hline 
\small{\textbf{Antworttyp}} & \multicolumn{2}{p{0.85\threecelltabwidth}|}{\inlinecode{JSON}} \\
\hline 
\end{tabular} 
\caption{Webservice Antworten (/answer)}
\end{table}

\subsubsection{Beispiel GET:}

\inlinecode{GET http://kort.herokuapp.com/server/webservices/answer/missing\_track\_type}

\textbf{Antwort:}

\lstset{language=JavaScript}
\begin{lstlisting}[style=examples]
{
	"return": [
		{
			"id":"1",
			"value":"grade1",
			"title":"Asphalt, Beton oder Pflastersteine",
			"sorting":"110",
			"type":"missing_track_type"
		},
		{ ... }
	]
}
\end{lstlisting}


% /bug
\subsection{Webservice: Fehler \emph{/bug}}
Die Fehler werden über den \inlinecode{/bug}-Webservice geladen.
Diesem muss die aktuelle Position mitgeliefert werden.
Dadurch ist es möglich lediglich die Fehler in der Umgebung zu laden.

Zusätzlich kann über diesen Webservice eine Lösung zu einem Fehler eingetragen werden.

\begin{table}[H]
\centering
\begin{tabular}{|p{0.15\threecelltabwidth}|p{0.25\threecelltabwidth}|p{0.6\threecelltabwidth}|}
\hline 
\small{\textbf{URL}} & \multicolumn{2}{p{0.85\threecelltabwidth}|}
{
\inlinecode{http://kort.herokuapp.com/server/webservices/bug/position/ <lat>,<lng>}
\newline \newline
\inlinecode{<lat>} Latitude der aktuellen Position 
\newline
\inlinecode{<lng>} Longitude der aktuellen Position
} \\ 
\hline 
\small{\textbf{Methode}} & \multicolumn{2}{p{0.85\threecelltabwidth}|}{\inlinecode{GET}} \\ 
\hline 
\small{\textbf{Parameter}} & \multicolumn{2}{p{0.85\threecelltabwidth}|}
{
\inlinecode{limit} Maximale Anzahl der zu ladenden Fehler \newline
\inlinecode{radius} Radius in dem sich die Fehler befinden müssen
} \\ 
\hline 
\small{\textbf{Antwort}} & \inlinecode{200 OK} & 
Daten konnten erfolgreich geladen werden. \\
\hline 
\small{\textbf{Antworttyp}} & \multicolumn{2}{p{0.85\threecelltabwidth}|}{\inlinecode{JSON}} \\
\hline 
\end{tabular} 
\caption{Webservice Fehler (GET /bug)}
\end{table}

\begin{table}[H]
\centering
\begin{tabular}{|p{0.15\threecelltabwidth}|p{0.25\threecelltabwidth}|p{0.6\threecelltabwidth}|}
\hline 
\small{\textbf{URL}} & \multicolumn{2}{p{0.85\threecelltabwidth}|}
{
\inlinecode{http://kort.herokuapp.com/server/webservices/bug/position/fix}
} \\ 
\hline 
\small{\textbf{Methode}} & \multicolumn{2}{p{0.85\threecelltabwidth}|}{\inlinecode{POST}} \\ 
\hline 
\small{\textbf{Parameter}} & \multicolumn{2}{p{0.85\threecelltabwidth}|}
{Die zu sendende Antwort muss als \inlinecode{JSON}-Objekt im Body gesendet werden.} \\ 
\hline 
\small{\textbf{Antwort}} & \inlinecode{200 OK} & 
Daten konnten erfolgreich geladen werden. Als Antwort werden die erspielten Punkte und Auszeichnungen gesendet. \\
\hhline{~--} & \inlinecode{403 Forbidden} & 
Der Benutzer ist nicht korrekt eingeloggt und kann somit keine Daten an den Server senden. \\
\hhline{~--} & \inlinecode{400 Bad request} & 
Das gesendete JSON ist nicht valide oder die Lösung konnte aus einem anderen Grund nicht in die Datenbank geschrieben werden. \\
\hline 
\small{\textbf{Antworttyp}} & \multicolumn{2}{p{0.85\threecelltabwidth}|}{\inlinecode{JSON}} \\
\hline 
\end{tabular} 
\caption{Webservice Fehler (POST /bug)}
\end{table}

\subsubsection{Beispiel GET:}

\inlinecode{GET http://kort.herokuapp.com/server/webservices/bug/position/47.1,8.1?\\limit=1\&radius=5000}

\textbf{Antwort:}

\lstset{language=JavaScript}
\begin{lstlisting}[style=examples]
{
	"return": [
		{
			"id":"32621371",
			"schema":"95",
			"type":"missing_track_type",
			"osm_id":"119068810",
			"osm_type":"way",
			"title":"Typ des Wegs unbekannt",
			"description":"Um welchen Weg-Typ handelt es sich hier?",
			"latitude":"47.0995850000000000",
			"longitude":"8.0979140000000000",
			"view_type":"select",
			"answer_placeholder":"Typ",
			"fix_koin_count":"5",
			"txt1":"",
			"txt2":"",
			"txt3":"",
			"txt4":"",
			"txt5":""
		},
		{ ... }
	]
}
\end{lstlisting}

\subsubsection{Beispiel POST:}

\inlinecode{POST http://kort.herokuapp.com/server/webservices/bug/fix}
\lstset{language=JavaScript}
\begin{lstlisting}[style=examples]
{
	"id":"ext-record-230",
	"user_id":3,
	"error_id":"28704192",
	"schema":"95",
	"osm_id":1611867263,
	"message":"McDonalds"
}
\end{lstlisting}

\textbf{Antwort:}

\lstset{language=JavaScript}
\begin{lstlisting}[style=examples]
{
	"badges": [
		{
			"name": "highscore_place_1"
		}
	],
	"koin_count_new":"15",
	"koin_count_total":"55"
}
\end{lstlisting}


% /highscore
\subsection{Webservice: Highscore \emph{/highscore}}
Über den Highscore-Webservice können die Benutzer sortiert nach Anzahl \emph{koins} geladen werden.

\begin{table}[H]
\centering
\begin{tabular}{|p{0.15\threecelltabwidth}|p{0.25\threecelltabwidth}|p{0.6\threecelltabwidth}|}
\hline 
\small{\textbf{URL}} & \multicolumn{2}{p{0.85\threecelltabwidth}|}
{
\inlinecode{http://kort.herokuapp.com/server/webservices/highscore}
} \\ 
\hline 
\small{\textbf{Methode}} & \multicolumn{2}{p{0.85\threecelltabwidth}|}{\inlinecode{GET}} \\ 
\hline 
\small{\textbf{Parameter}} & \multicolumn{2}{p{0.85\threecelltabwidth}|}{\inlinecode{limit} Maximale Anzahl der Benutzer} \\ 
\hline 
\small{\textbf{Antwort}} & \inlinecode{200 OK} & 
Daten konnten erfolgreich geladen werden. \\
\hline 
\small{\textbf{Antworttyp}} & \multicolumn{2}{p{0.85\threecelltabwidth}|}{\inlinecode{JSON}} \\
\hline 
\end{tabular} 
\caption{Webservice Antworten (/highscore)}
\end{table}

\subsubsection{Beispiel GET:}

\inlinecode{GET http://kort.herokuapp.com/server/webservices/highscore?limit=10}

\textbf{Antwort:}

\lstset{language=JavaScript}
\begin{lstlisting}[style=examples]
{
	"return": [
		{
			"user_id":"3",
			"username":"tschortsch",
			"koin_count":"140",
			"fix_count":"12",
			"vote_count":"4",
			"ranking":"1",
			"you":true
		},
		{ ... }
	]
}
\end{lstlisting}


% /osm
\subsection{Webservice: OpenStreetMap \emph{/osm}}
Um OpenStreetMap-Objekte auf der Karte anzuzeigen, werden über den \inlinecode{/osm}-Webservice die entsprechenden OSM-Daten geladen.
Der Webservice leitet den Request an das OSM API\footnote{\url{http://wiki.openstreetmap.org/wiki/API_v0.6}} weiter und sendet das Resultat an die Webapplikation zurück.

\begin{table}[H]
\centering
\begin{tabular}{|p{0.15\threecelltabwidth}|p{0.25\threecelltabwidth}|p{0.6\threecelltabwidth}|}
\hline 
\small{\textbf{URL}} & \multicolumn{2}{p{0.85\threecelltabwidth}|}
{
\inlinecode{http://kort.herokuapp.com/server/webservices/osm/<type>/<id>} 
\newline \newline
\inlinecode{<type>} OSM-Objekttyp
\newline
\inlinecode{<id>} ID des OSM-Objekts
} \\ 
\hline 
\small{\textbf{Methode}} & \multicolumn{2}{p{0.85\threecelltabwidth}|}{\inlinecode{GET}} \\ 
\hline 
\small{\textbf{Parameter}} & \multicolumn{2}{p{0.85\threecelltabwidth}|}{-} \\ 
\hline 
\small{\textbf{Antwort}} & \inlinecode{200 OK} & 
Daten konnten erfolgreich geladen werden. \\
\hline 
\small{\textbf{Antworttyp}} & \multicolumn{2}{p{0.85\threecelltabwidth}|}{\inlinecode{XML}} \\
\hline 
\end{tabular} 
\caption{Webservice OpenStreetMap (/osm)}
\end{table}

\subsubsection{Beispiel GET:}

\inlinecode{GET http://kort.herokuapp.com/server/webservices/osm/node/1658024260}

\textbf{Antwort:}

\lstset{language=XML}
\begin{lstlisting}[style=examples]
<?xml version="1.0" encoding="UTF-8"?>
<osm version="0.6" generator="OpenStreetMap server" copyright="OpenStreetMap and contributors" attribution="http://www.openstreetmap.org/copyright" license="http://opendatacommons.org/licenses/odbl/1-0/">
  <node id="1658024260" version="1" changeset="10861664" lat="47.5114378" lon="8.5443127" user="pfrauenf" uid="479871" visible="true" timestamp="2012-03-03T20:05:48Z">
    <tag k="amenity" v="fast_food"/>
  </node>
</osm>
\end{lstlisting}


% /user
\subsection{Webservice: Benutzer \emph{/user}}
Der Benutzer-Webservice dient zur Authentifizierung des Benutzers.
Über ihn können sich die Benutzer an- und abmelden.
Zudem werden die Benutzerdaten über ihn geladen.

\begin{table}[H]
\centering
\begin{tabular}{|p{0.15\threecelltabwidth}|p{0.25\threecelltabwidth}|p{0.6\threecelltabwidth}|}
\hline 
\small{\textbf{URL}} & \multicolumn{2}{p{0.85\threecelltabwidth}|}
{
\inlinecode{http://kort.herokuapp.com/server/webservices/user/[<secret>]} 
\newline \newline
\inlinecode{<secret>} (optional) User Secret wird gesendet falls der Benutzer bereits eingeloggt ist.
} \\ 
\hline 
\small{\textbf{Methode}} & \multicolumn{2}{p{0.85\threecelltabwidth}|}{\inlinecode{GET}} \\ 
\hline 
\small{\textbf{Parameter}} & \multicolumn{2}{p{0.85\threecelltabwidth}|}{-} \\ 
\hline 
\small{\textbf{Antwort}} & \inlinecode{200 OK} & 
Daten konnten erfolgreich geladen werden. Der Webservice liefert die Benutzerdaten zurück. \\
\hline 
\small{\textbf{Antworttyp}} & \multicolumn{2}{p{0.85\threecelltabwidth}|}{\inlinecode{JSON}} \\
\hline 
\end{tabular} 
\caption{Webservice Benutzer (/user)}
\end{table}

\subsubsection{Beispiel GET:}

\inlinecode{GET http://kort.herokuapp.com/server/webservices/user}

\textbf{Antwort:}

\lstset{language=JavaScript}
\begin{lstlisting}[style=examples]
{
	"return": {
		"id":"3",
		"name":"J\u00fcrg Hunziker",
		"username":"tschortsch",
		"oauth_user_id":"email@host.com",
		"oauth_provider":"Google",
		"token":null,
		"fix_count":"2",
		"vote_count":"4",
		"koin_count":"40",
		"secret":"secret",
		"pic_url":"http:\/\/www.gravatar.com\/avatar\/secret?s=200&d=mm&r=r",
		"logged_in":true
	}
}
\end{lstlisting}

\subsubsection{Beispiel GET /id/badges:}

\inlinecode{GET http://kort.herokuapp.com/server/webservices/user/<id>/badges}

\textbf{Antwort:}

\lstset{language=JavaScript}
\begin{lstlisting}[style=examples]
{
	"return": [
		{
			"id":"1",
			"name":"highscore_place_1",
			"title":"1. Rang",
			"description":"Erster Rang in der Highscore erreicht.",
			"color":"#FFFBCB",
			"sorting":"110",
			"won":true,
			"create_date":"13.12.2012 18:56"
		}
	]
}
\end{lstlisting}

\todo[inline]{/user Webservice komplett dokumentieren}
\todo[inline]{/db + /validation Webservice dokumentieren}