\section{Implementation}
\label{backend-implementation}
Das Backend besteht neben der Datenbank und den \gls{REST}-Schnittstellen vor allem aus PHP-Code.
Der meiste Code entfällt auf das Handling von Webservice-Anfragen und die Authentifizierung mit \gls{OAuth}.

\subsection{Gliederung}
\label{backend-gliederung}
Das Backend befindet sich im Verzeichnis \inlinecode{server/} im Repository.
Das Backend teilt sich auf verschiedene Unterordner auf (siehe Tabelle \ref{table-backend-gliederung}).

\begin{table}[H]
\centering
\begin{tabular}{|p{0.25\twocelltabwidth}|p{0.75\twocelltabwidth}|}
\hline 
\textbf{Order} & \textbf{Inhalt} \\
\hline 
\inlinecode{database/} & SQL und Shell-Skripts für die Erstellung der Datenbank \\
\hline 
\inlinecode{heroku/} & Shell-Skripte für das Deployment auf Heroku \\
\hline 
\inlinecode{oauth2callback/} & Callback-Handler der verschiedenen \gls{OAuth}-Dienste \\
\hline 
\inlinecode{php/} & PHP-Klassen für das Backend \\
\hline 
\inlinecode{redmine/} & Skripts und Anleitung für Redmine \\
\hline 
\inlinecode{ssh\_pub\_keys/} & Öffentliche SSH-Schlüssel für das Deployment \\
\hline 
\inlinecode{webservices/} & \gls{REST}-Ressourcen (Endpunkte der Schnittstellen) \\
\hline 
\end{tabular}
\caption{Gliederung des Backends}
\label{table-backend-gliederung}
\end{table}

Speziell zu erwähnen sind dabei die PHP-Klassen welche die Logik des Backends abbilden.
Sie sind über Namespaces aufgeteilt und liefern die Logik für alle anderen Teile des Backends.
Um dies zu ermöglichen gibt es die \inlinecode{ClassLoader}-Klasse\footnote{\url{http://kort.herokuapp.com/docs/Kort-backend/classes/Kort.ClassLoader.html}}.
Falls irgendwo eine PHP-Klasse gebraucht wird, muss nur diese Klasse geladen werden, diese wiederum kümmert sich darum, alle abhängigen Datei nachzuladen.

Für die Klassen gibt es eine separate Dokumentation (siehe Abschnitt \ref{backend-dokumentation}).

\subsection{Abhängigkeiten}
\label{backend-abhaengigkeiten}

\begin{table}[H]
\centering
\begin{tabular}{|p{0.35\threecelltabwidth}|p{0.15\threecelltabwidth}|p{0.50\threecelltabwidth}|}
\hline 
\textbf{Library} & \textbf{Version} & \textbf{Verwendung} \\
\hline 
Slim & 2.1.0 & Micro-Framework für die Implementation von \gls{REST}-Schnittstellen \\
\hline 
Google APIs Client Library & 0.6.0 & PHP-Library für Google \glspl{API} \\
\hline 
oauth-php & 175 & \gls{OAuth} Library für PHP \\
\hline 
Ant-Contrib & 1.0b3 & Erweiterte Tasks für Apache Ant \\
\hline 
\end{tabular}
\caption{Abhängigkeiten im Backend}
\label{table-backend-dependencies}
\end{table}

