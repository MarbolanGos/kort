\chapter{Fehler-Datenquellen}
\label{datenquellen}

\todo[inline]{Fehlerdatenquellen überarbeiten}

Die Grundlage unserer \gls{WebApp} bilden die Fehlerdaten, welche korrigiert werden sollen.
Es galt deshalb zuerst eine geeignete Quelle für Fehlerdaten in OpenStreetMap zu finden.

\section{Evaluation von geeigneten Datenquellen}
Durch die Anforderungen an die Applikation ergaben sich folgende Bedingungen, welche die Fehlerdaten erfüllen müssen:

\begin{itemize}
\item Die Fehler müssen auch für Nicht-Mapper lösbar sein. Sprich die Beschreibungen sollten kein Mapping-Vokabular enthalten.
\item Sie müssen auf Deutsch übersetzbar sein. Die Fehlerbeschreibungen sollten somit einem gegebenen Schema folgen und nicht Freitext beinhalten.
\item Dadurch dass wir eine mobile App erstellen und keinen vollwertigen Karten-Editor, sollen die Fehler lediglich Tags betreffen. Das Lösen des Fehlers sollte also möglich sein, ohne das Geometrieobjekt (Strasse, Gebäude, usw.) zu verändern.
\item Die Fehler sollten sich immer nur auf einen Tag des betroffenen OSM-Objekt beziehen. Dadurch vereinfacht sich das Zurückschreiben der Daten zu OpenStreetMap.
\end{itemize}

Wir untersuchten deshalb mehrere Fehlerdatenquellen auf deren Qualität.

\subsection{FIXME-Tags in OpenStreetMap}
Die Objekte in OpenStreetMap können von Benutzern direkt mit einem FIXME-Tag versehen werden, wenn sie einen Fehler aufweisen.

\begin{table}[H]
\centering
\begin{tabular}{|p{0.25\twocelltabwidth}|p{0.75\twocelltabwidth}|}
\hline 
\small{\textbf{URL}} & \url{http://wiki.openstreetmap.org/wiki/DE:Key:fixme} \\
\hline 
\small{\textbf{Erfasser}} & OpenStreetMap-Benutzer \\
\hline 
\small{\textbf{API verfügbar?}} & Ja (API oder Datenbankzugriff) \\
\hline 
\small{\textbf{Lösung zurückschreiben}} & Lösung könnte via API an OSM gesendet werden. \\
\hline
\small{\textbf{Eignung}} & \textbf{tief} \linebreak Durch die manuelle Erfassung der Daten, ist es für uns nicht möglich ein spezifisches GUI für die Fehlerbehebung zu erstellen. \\ 
\hline 
\end{tabular} 
\caption{FIXME-Tags in OpenStreetMap}
\label{datenquellen-fixme_tags_osm}
\end{table}

\subsection{OpenStreetBugs}
Mit der Applikation OpenStreetBugs ist es für die Benutzer möglich Fehler in den Kartendaten als Bug zu erfassen.

\begin{table}[H]
\centering
\begin{tabular}{|p{0.25\twocelltabwidth}|p{0.75\twocelltabwidth}|}
\hline 
\small{\textbf{URL}} & \url{http://openstreetbugs.schokokeks.org/} \\
\hline 
\small{\textbf{Erfasser}} & OpenStreetBugs-Benutzer \\
\hline 
\small{\textbf{API verfügbar?}} & ??????? \\
\hline 
\small{\textbf{Lösung zurückschreiben}} & Lösung müsste an OpenStreetMaps gesendet werden. Zusätzlich  müsste der Bug auf OpenStreetBugs als gelöst markiert werden. \\
\hline
\small{\textbf{Eignung}} & \textbf{tief} \linebreak Durch die manuelle Erfassung der Daten, ist es für uns nicht möglich ein spezifisches GUI für die Fehlerbehebung zu erstellen. \\ 
\hline 
\end{tabular} 
\caption{OpenStreetBugs}
\label{datenquellen-openstreetbugs}
\end{table}

\subsection{Keepright}
keepright ist ein Dienst, welcher jeweils über Nacht, Fehler in OSM-Objekten sucht und in einer eigenen Datenbank ablegt.

\begin{table}[H]
\centering
\begin{tabular}{|p{0.25\twocelltabwidth}|p{0.75\twocelltabwidth}|}
\hline 
\small{\textbf{URL}} & \url{http://keepright.ipax.at} \\
\hline 
\small{\textbf{Erfasser}} & Automatisiert durch nächtlichen Job \\
\hline 
\small{\textbf{API verfügbar?}} & Fehlerdaten werden als Dump-File zum Download angeboten. \\
\hline 
\small{\textbf{Lösung zurückschreiben}} & Lösung müssten lediglich an OpenStreetMap gesendet werden. keepright baut seine Fehlerdatenbank jeweils über Nacht aus dem OSM-Daten neu auf. \\
\hline
\small{\textbf{Eignung}} & \textbf{hoch} \linebreak Dadurch das die Fehlerdaten automatisiert erstellt werden, ist es gut möglich, für jeweilige Fehlertypen eine geeignete Maske für deren Behebung zur Verfügung zu stellen. \\
\hline
\end{tabular}
\caption{keepright}
\label{datenquellen-keepright}
\end{table}

\section{Entscheid: Fehlerdaten von keepright}
Durch die gute Eignung der Fehlerdaten vom keepright-Dienst, haben wir uns dazu entschieden deren Daten als Basis für unsere Fehler zu verwenden.

\section{Fehlerdaten in die Datenbank laden}
\todo[inline]{Fehlerdaten laden (Shell-Skripts etc.) beschreiben}

\subsection{Datenformat}