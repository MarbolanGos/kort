\chapter{Backend-Administration}
\label{backend_administration}

\section{Hinzufügen von Fehler-Datenquellen}

\section{Hinzufügen von Fehlertypen}

\subsection{Fehlertypen}
In der App sind bereits folgende Fehlertypen implementiert:

\begin{itemize}
\item Autobahn ohne Bezeichner
\item Kultstätte/Kirche ohne Religion
\item Objekt ohne Namen
\item Beziehung ohne Typ
\item Fehlendes Tempolimit
\item Sprache des Namens unbekannt
\item Typ des Wegs unbekannt
\item Strasse ohne Namen
\end{itemize}

Diese Typen sind in der Tabelle \inlinecode{keepright.error\_type} definiert.
Im Feld \inlinecode{type} dieser Tabelle befindet sich die eindeutige Identifikation eines Typs.
Diese wird beispielsweise für die Anzeige eines passenden Marker-Icons und der Verbindung zu einem passenden View-Typ (siehe Abschnitt \ref{view_types}) verwendet.

% IMAGE keepright.error_type

\subsubsection{Erstellen eines neuen Fehlertyps}


\subsection{View-Typen}
\label{view_types}
All diese Typen werden dann wiederum einem spezifischen View-Typen zugeordnet.
Dieser bestimmt, wie das Formular zum Lösen des Fehlers in der Benutzeroberfläche aussieht.

In Tabelle \ref{kort-view-types-table} sind die bereits vorhandenen View-Typen beschrieben.

\begin{table}[H]
\centering
\begin{tabular}{|p{0.15\twocelltabwidth}|p{0.85\twocelltabwidth}|}
\hline
\textbf{Typ} & \textbf{Beschreibung} \\
\hline
text & Rendert ein Text-Eingabefeld \\
\hline
number & Rendert ein Zahl-Eingabefeld \\
\hline
select & Rendert eine Select-Box mit vorgefüllten Werten aus der Datenbank \\
\hline
\end{tabular}
\caption{kort: View-Typen}
\label{kort-view-types-table}
\end{table}

Wird als View-Typ \emph{select} gewählt, müssen in der Tabelle \inlinecode{kort.answer} noch die möglichen Antworten eingetragen werden.
Darin kann der eigentliche Wert des OpenStreetMap-Tags und eine passende Bezeichnung hinterlegt werden.
Als \inlinecode{type} muss der jeweilige Typen-Bezeichner gewählt werden.

\subsubsection{Erstellen eines neuen View-Typen}
Um einen neuen View-Typen zu definieren muss wie in Code-Ausschnitt \ref{kort-add_view_type} beschrieben die Unterscheidung in der Klasse \inlinecode{Kort.view.bugmap.fix.Form} um den neuen Typen ergänzt werden.

\lstset{language=JavaScript}
\begin{lstlisting}[float, caption=Hinzufügen eines View-Typen in der Klasse Kort.view.bugmap.fix.Form, label=kort-add_view_type]
createFixField: function(bug) {
	[...]
	
	if(bug.get('view_type') === '<Neuer View-Typ>') {
		fixField = Ext.create('<Neue View-Klasse>', fieldConfig);
	} else {
		...
	}
	
	[...]
}
\end{lstlisting}

\section{Hinzufügen von Auszeichnungen}
\label{kort-additional-badges}
Die bereits vorhandenen Auszeichnungen sind in Tabelle \ref{kort-badges} beschrieben. Um weitere Auszeichnungen hinzuzufügen, muss folgendermassen vorgegangen werden:

\begin{enumerate}
\item Es muss ein neuer Badge in der Tabelle \inlinecode{kort.badge} erstellt werden
\item Zusätzlich muss die Regel für das Gewinnen des Badges in der Abfrage XY definiert werden. \todo[inline]{Abfrage erstellen und einfügen}
\item Für das Frontend muss ein Bild erstellt werden, welches dem Namen (Tabellenattribut \inlinecode{kort.badge.name}) des Badges entspricht. Dieses Bild muss in folgendem Pfad gespeichert werden \inlinecode{/resources/images/badges/<badgename>.png}.
\end{enumerate}

Sind alle Punkte abgearbeitet ist der Badge im Frontend ersichtlich und kann von den Benutzern gewonnen werden.