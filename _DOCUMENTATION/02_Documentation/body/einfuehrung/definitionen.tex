\chapter{Begriffsdefinitionen}
\label{kort-definitionen}

Für diese Arbeit wurden einige neue Begriffe verwendet und Konzepte mit einem Namen definiert.
Dieses Kapitel soll kurz die wichtigsten Begriffe von \kort erklären.
Hierbei handelt es sich um fachliches Vokabular, welches in dieser Arbeit verwendet wird.
Alle anderen, eher technischen Begriffe, befinden sich im Glossar (siehe Teil \ref{glossar}).

\section{Name der Applikation}
Beim Begriff \kort handelt es sich um einen skandinavischen Ausdruck für \emph{Karte}.
Obwohl dieser Begriff sehr allgemein ist, scheint er im Zusammenhang mit Applikationen noch nicht häufig verwendet worden zu sein.
Der Name ist kurz und prägnant, was für eine App ideal ist.

Falls die App zukünftig auch im skandinavischen Raum verwendet wird, muss eine Namensänderung allfällig in Betracht gezogen werden.

\section{Begriffe aus dem Spiel}

\begin{table}[H]
\centering
\begin{tabular}{|p{0.2\threecelltabwidth}|p{0.12\threecelltabwidth}|p{0.68\threecelltabwidth}|}
\hline 
\small{\textbf{Kategorie}} & \small{\textbf{Begriff}} & \small{\textbf{Beschreibung}} \\
\hline 
Spieleinheit & Auftrag & Aufträge sind Fehler auf der Karte, welche von einem Benutzer korrigiert werden. \\
\hline 
Belohnung & Koins & Punkte, welcher ein Benutzer gewinnen kann, werden \emph{Koins} genannt.
Das Wort ist vom Englischen \emph{coin} (Münze) abgeleitet. 
Das "`k"' ist dabei eine Anlehnung an den Spielnamen \kort. \\
\hline 
Auszeichnungen & Badge & Die Auszeichnungen, die ein Benutzer gewinnen kann werden \emph{Badge} genannt. \\
\hline 
Rangliste & Highscore & Die Rangliste aller Spieler sortiert nach Anzahl \emph{Koins}. \\
\hline 
Punkt auf der Karte & Objekt & Ein speziell ausgezeichneter Ort (\gls{POI}) auf der Karte zu dem Informationen fehlen. \\
\hline 
\end{tabular}
\caption{Verschiedene Begriffe aus \kort}
\label{table-definitionen}
\end{table}