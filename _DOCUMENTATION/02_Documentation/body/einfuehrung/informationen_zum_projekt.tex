\chapter{Informationen zum Projekt}
\label{informationen-projekt}


\section{Problemstellung}
\brand{OpenStreetMap} ist ein freies Projekt, welches jedermann ermöglicht, Kartendaten anzuzeigen und zu editieren.
Durch diesen öffentlichen Charakter ist es nicht ausgeschlossen, dass fehlerhafte bzw. unvollständige Daten eingetragen werden.

Aus diesem Grund entstand die Idee, mit einer mobilen Applikation eine Möglichkeit anzubieten, Fehler auf einfache Weise anzuzeigen und zu korrigieren.
Um die Motivation für die Verwendung der App längerfristig aufrecht zu erhalten, soll diese mit Spiel-Elementen versehen werden.

\section{Aufgabenstellung}
Im Rahmen dieser Arbeit wird eine HTML5 \gls{WebApp} entwickelt, welche es ermöglicht, unvollständige bzw. fehlerhafte Daten in der \brand{OpenStreetMap}-Datenbank zu korrigieren oder zu vervollständigen.

Die App soll nicht als herkömmlicher Editor implementiert werden, sondern einen gewissen Gamecharakter aufweisen.
Dieser zeichnet sich dadurch aus, dass die Benutzer für ihre Änderungsvorschläge belohnt werden.
So können sie beispielsweise in der Rangliste (Highscore) aufsteigen oder Auszeichnungen (Badges) gewinnen.
Dieses Konzept ist unter dem Stichwort \gls{Gamification} bekannt.
Damit ist allgemein die Anwendung von Spiel-Elementen in einem spielfremden Kontext gemeint.
Es soll untersucht werden, ob und wie sich dies für \brand{OpenStreetMap} umsetzen lässt.

Eingetragene Änderungsvorschläge können anschliessend von weiteren Benutzern kontrolliert und bewertet werden.
Mehrfach validierte Änderungen sollen ins \brand{OpenStreetMap}-Projekt zurückgeführt werden.

Mit der Integration von Social Media-Diensten (Facebook, Twitter) soll die Bekanntheit der App gefördert und die Motivation der Benutzer gesteigert werden.

\section{Ziele}
In der Aufgabenstellung der Arbeit wurden folgende Ziele definiert:
\begin{itemize}
	\item Erstellen einer cross-platform HTML5 \gls{WebApp} mit JavaScript
	\item Einsatz von JavaScript-\glspl{API} zur Ansteuerung von Hardwarekomponenten (z.B. GPS, Kamera)
	\item Es soll geprüft werden, ob die \gls{WebApp} auch als native App für die Plattformen iOS und Android zur Verfügung gestellt werden kann 
	\item Als Basis sollen Daten und Webdienste des \brand{OpenStreetMap}-Projekts verwendet werden
	\begin{itemize}
		\item Quellen für bekannte Fehler: \brand{OpenStreetMap} Bug Reports, \inlinecode{FIXME}- und \inlinecode{TODO}-\glspl{Tag}
		\item Unterstützung von \gls{POI}-, Linestring- und Polygon-Objekten
	\end{itemize}
	\item Verwendung einer vorhandenen User-Basis für die Authentifizierung (\gls{OAuth})
	\item Integration von Social Media zum Austausch von Aktivitäten
	\item Konzept für Gamification von \brand{OpenStreetMap} erarbeiten
	\begin{itemize}
		\item Highscores / Rankings
		\item Badges / Achievements
		\item Aufgaben mit verschiedenen Schwierigkeitsstufen
	\end{itemize}
	\item Verschiedene Modi
	\begin{itemize}
		\item Erfassen von Daten, Aufnahme von Fotos (ortsbezogen)
		\item Verifikation von eingegebenen Daten (ortsunabhängig)
	\end{itemize}
	\item Das User Interface soll primär auf Deutsch erstellt werden. Es sollen jedoch Vorkehrungen getroffen werden, um Übersetzungen einfach zu ermöglichen (Internationalisierung)
\end{itemize}

\section{Rahmenbedingungen}
\begin{itemize}
\item Es gelten die Rahmenbedingungen, Vorgaben und Termine der HSR
\item Die Projektabwicklung orientiert sich an einer iterativen, agilen Vorgehensweise. Als Vorgabe dient dabei Scrum, wobei bedingt durch das kleine Projektteam gewisse Vereinfachungen vorgenommen werden. Meilensteine werden bezüglich Termin und Inhalt mit dem Betreuer vereinbart.
\item Die Kommunikation in der Projektgruppe, in der Dokumentation und an den Präsentationen erfolgt auf Deutsch.
\end{itemize}

\section{Aufbau der Arbeit}
Die Arbeit ist in drei Teile gegliedert. Im ersten Teil erfolgt eine Einleitung mit allgemeinen Informationen zum Projekt und dessen Umsetzung (siehe Kapitel \ref{informationen-projekt}. Informationen zum Projekt, \ref{umsetzung}. Umsetzung und \ref{kort-definitionen}. Definitionen). Darin werden auch Begriffe erklärt, welche während der Arbeit entwickelt oder oft verwendet werden.

Der zweite Teil beinhaltet die Dokumentation der eigentlichen Arbeitsresultate (siehe Kapitel \ref{kort}. Kort, \ref{leaflet-sencha-komponente}. Leaflet Komponente, \ref{gamification}. Gamification, \ref{datenquellen}. Datenquellen, \ref{architektur}. Architektur, \ref{infrastruktur}. Infrastruktur, \ref{backend}. Backend und \ref{administration}. Administration). Darin wird unter anderem die Implementation der \gls{WebApp} beschrieben und die Infrastruktur, welche für den Betrieb notwendig ist.

Im dritten Teil befinden sich Informationen zum Projektmanagement (siehe Kapitel \ref{projektmanagement}. Projektmanagement und \ref{projektmonitoring}. Projektmonitoring).

Neben diesem Dokument umfasst die Arbeit die implementierte \gls{WebApp} \kort{}. Der dazugehörige Source Code ist frei im Internet zugänglich, sowie auf der beigelegten CD zu finden.

\begin{table}[H]
\centering
\begin{tabular}{|p{0.3\twocelltabwidth}|p{0.7\twocelltabwidth}|}
\hline 
\textbf{Arbeitsresultat} & \textbf{URL} \\ 
\hline 
\kort{} (\gls{WebApp}) & \url{http://kort.herokuapp.com} \\ 
\hline 
Repository & \url{https://github.com/odi86/kort} \\ 
\hline 
\end{tabular}
\label{arbeitsresultate}
\caption{Übersicht der Arbeitsresultate}
\end{table} 