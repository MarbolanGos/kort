\chapter{Projektmonitoring}
\label{projektmonitoring}

% Meilensteine
\section{Meilensteine}
\label{meilensteine}

Um den Projektverlauf zu beurteilen, haben wir folgende Meilensteine erstellt:

\subsubsection{Meilenstein 1: Aufbereiten von Fehlerdaten}
\tick Geeignete Fehler finden \\
\tick Fehler in Datenbank speichern \\
\tick Fehler filtern \\
\tick Fehler zugänglich machen

\subsubsection{Meilenstein 2: Lesen von Fehlerdaten}

\tick Zugriff auf Datenbank aus App \\
\tick Persistieren der Daten in App

\subsubsection{Meilenstein 3: Kartenanzeige von Fehlern}

\tick Geeignete Kartendarstellung von Fehler finden \\
\tick Fehler auf der Karte anzeigen \\
\tick Location-based Karte

\subsubsection{Meilenstein 4: OAuth}

\tick Login möglich über OAuth \\
\tick User wird in App persistiert \\
\tick Logout möglich \\
\cross Login mit OSM-OAuth

\subsubsection{Meilenstein 5: Schreiben von Fehlerbehebungen}

\tick Input von User speichern in DB \\
\tick Unterscheidung von Fehlertypen

\subsubsection{Meilenstein 6: Verifizieren von Änderungen}

\tick UI zum verifizieren von Fehlerbehebungen \\
\cross Implementation eines Thresholds für Verfikationen

\subsubsection{Meilenstein 7: Daten zu OSM schicken}

\cross Verifikationsdaten einheitlich via API an OSM senden \\
\cross Prüfung ob eine Änderung zulässig ist

\subsubsection{Meilenstein 8: Gamification-Konzepte (Highscore, Leaderboard, Badges, Achievements)}

\cross Erarbeitung eines Gamifications-Konzepts für OSM \\
\tick Geeignete Elemente in App umsetzen (\tick Highscore, \tick Badges)

% Risikomanagement
\section{Risikomanagement}
\label{risikomanagement}

Für das Projekt wurden folgende Risiken identifiziert:
\subsection{Technische Risiken}

\subsubsection{OpenStreetMap-Daten können nicht in einer Sencha Touch-Applikation angezeigt werden}
\begin{table}[H]
\centering
\begin{tabular}{|p{0.25\twocelltabwidth}|p{0.75\twocelltabwidth}|}
\hline 
\small{\textbf{Auswirkung}} & Es müsste ein alternatives mobiles Framework gefunden werden, welches mit \brand{OpenStreetMap}-Daten umgehen kann. \\
\hline 
\small{\textbf{Wahrscheinlichkeit}} & tief \\
\hline 
\small{\textbf{Massnahme zur Verhinderung}} & Zu Beginn des Projektes muss eine Sencha Touch Prototyp-Applikation implementiert werden, welche \brand{OpenStreetMap}-Daten auf der Karte darstellt. \\
\hline
\end{tabular}
\end{table}

\subsubsection{Keine geeignete Fehlerdatenquelle vorhanden}
\begin{table}[H]
\centering
\begin{tabular}{|p{0.25\twocelltabwidth}|p{0.75\twocelltabwidth}|}
\hline 
\small{\textbf{Auswirkung}} & Es müsste eine Möglichkeit gefunden werden, vorhandene Fehlerdaten so abzuändern, dass sie für den Einsatz in der App verwendet werden können. \\
\hline 
\small{\textbf{Wahrscheinlichkeit}} & mittel \\
\hline 
\small{\textbf{Massnahme zur Verhinderung}} & Es muss eine Evaluation von bestehenden Fehlerdatenquellen durchgeführt werden. \\
\hline
\end{tabular}
\end{table}

\subsubsection{Schwierigkeiten mit OAuth}
\begin{table}[H]
\centering
\begin{tabular}{|p{0.25\twocelltabwidth}|p{0.75\twocelltabwidth}|}
\hline 
\small{\textbf{Auswirkung}} & OAuth ist als Protokoll bekannt, welches Schwierigkeiten bereiten kann.
Falls sich die Anforderungen nicht umsetzen lassen, muss eine alternative Lösung gefunden werden für den Login. \\
\hline 
\small{\textbf{Wahrscheinlichkeit}} & mittel \\
\hline 
\small{\textbf{Massnahme zur Verhinderung}} & Wir müssen genügend Zeit für OAuth einplanen. \\
\hline
\end{tabular}
\end{table}

\subsection{Weitere Risiken}

\subsubsection{OpenStreetMap erlaubt keinen allgemeinen Benutzer zum Zurückschreiben der Fehlerbehebungen}
\begin{table}[H]
\centering
\begin{tabular}{|p{0.25\twocelltabwidth}|p{0.75\twocelltabwidth}|}
\hline 
\small{\textbf{Auswirkung}} & Es müsste eine Möglichkeit gefunden werden, die Fehlerbehebungen trotzdem in \brand{OpenStreetMap} einpflegen zu können \\
\hline 
\small{\textbf{Wahrscheinlichkeit}} & mittel \\
\hline 
\small{\textbf{Massnahme zur Verhinderung}} & \brand{OpenStreetMap}-Community muss von der Idee hinter \kort{} überzeugt werden. \\
\hline
\end{tabular}
\end{table}

\subsubsection{Zu wenig Erfahrung mit Game-Design}
\begin{table}[H]
\centering
\begin{tabular}{|p{0.25\twocelltabwidth}|p{0.75\twocelltabwidth}|}
\hline 
\small{\textbf{Auswirkung}} & Die App hat keinen Game-Charakter oder wird vom Benutzer nicht als solches erkannt. \\
\hline 
\small{\textbf{Wahrscheinlichkeit}} & gross \\
\hline 
\small{\textbf{Massnahme zur Verhinderung}} & Unser Industriepartner hat viel Erfahrung mit der Entwicklung von Games und kann uns mit Ratschlägen unterstützen.
Daneben haben wir versucht, Designer zu involvieren welche uns unterstützen können. \\
\hline
\end{tabular}
\end{table}

\section{Projektverlauf}
Das Projekt bestand aus 5 Sprints, wobei wir ursprünglich 6 geplant haben.
Gegen Ende des Projekts mussten wir uns eingestehen, dass der Overhead für zweiwöchige Sprints zu gross ist, deshalb haben wir die letzten beiden Sprints auf 3 Wochen erweitert.

Dank \brand{Redmine}, unsere Projektmanagement-Tool, hatten wir eine gute Übersicht über das Projekt.
Jeweils am Ende eines Sprints haben wir besprochen wie wir weiterfahren sollen und haben entsprechend den nächsten Sprint geplant.
Durch diese Iterationen war es uns möglich, schnell ein System zu erstellen, welches den aktuellen Bedürfnissen entspricht.

Im Laufe des Projeks hat es sich ergeben, dass sich Herr Hunziker eher um das Frontend und Herr Oderbolz eher um das Backend gekümmert hat.
Die Grenzen sind dabei fliessend und sind grösstenteils durch die persönlichen Interessen entstanden.
Da wir das Studium berufsbegleitend absolvieren und uns deshalb nicht jeden Tag getroffen haben, hatte dies den Vorteil, dass wir so relativ gut unabhängig voneinander arbeiten konnten.

Etwa in der Hälfe des Projekts wollte unser Betreuer Prof. Stefan Keller wissen, welche Funktionalitäten wir bis zum Ende abschliessen können und welche nicht.
Die agile Vorgehensweise erlaubt es grundsätzlich nicht solche Aussagen zu treffen, da sich der Scope noch ändern kann.
Jedoch ist es natürlich verständlich, dass man eine Übersicht will wie das Projekt vorangeschritten ist.
Deshalb haben wir dann begonnen eine Meilenstein-Liste zu führen, welche die wichtigsten Funktionalitäten aufzeigt.
So konnten wir zum einen jeweils zeigen, was schon erledigt und was noch zu tun ist und zum anderen jeweils beim Sprint Planning direkt Meilensteine einplanen.

Abschliessend ist zu sagen, dass das Projekt gut verlaufen ist, und wir grundsätzlich alle unsere Ziele erreichen konnten (siehe Abschnitt \ref{fazit}).
Daneben gab es auch Raum um kreative Ideen auszuprobieren.

\section{Arbeitsaufwand}
Wie im Kapitel \ref{projektmanagement} beschrieben, war der vom Modul vorgegebene Aufwand pro Person auf \emph{320 Stunden} festgelegt. Leider haben wir beide diese Vorgabe leicht überschritten (siehe Tabelle \ref{projektmanagement-arbeitsaufwand}).
\todo[inline]{Beschreibung Arbeitsaufwand}

\begin{table}[H]
\centering
\begin{tabular}{|l|l|}
\hline 
\textbf{Person} & \textbf{Aufwand} \\ 
\hline 
Jürg Hunziker & 339h \\
\hline 
Stefan Oderbolz & 352h \\  
\hline 
\end{tabular}
\caption{Arbeitsaufwand pro Person}
\label{projektmanagement-arbeitsaufwand}
\end{table} 

\section{Fazit}
\label{fazit}
Bereits die Liste der Meilensteine (siehe Abschnitt \ref{meilensteine}) vermittelt ein ambitioniertes Ziel, welches zu erreichen war.
Leider konnten wir nicht alle Punkte abschliessen.
Schlussendlich haben wir uns auf die Grundfunktionalität der \gls{WebApp} konzentriert.
So mussten wir leider auf das Zurückschreiben der korrigierten Daten zu \gls{OpenStreetMap} verzichten.

Die entwickelte \gls{WebApp} erfüllt alle Erwartungen an eine moderne Applikation.
Ein Benutzer kann mit der App die gewünschten Aufgaben (Fehler korrigieren und validieren) durchführen und wird durch einige Gamification-Konzepte animiert die App weiter zu verwenden.

In der doch kurzen Zeit ist es uns aber nicht gelungen ein vollwertiges Spiel zu entwickeln.
Dazu müsste noch mehr Wert auf Details gelegt werden, so dass eine abgerundetes Spielerlebnis entsteht.

Dank der agilen Vorgehensweise konnten wir schnell auf Änderungen reagieren, welche sich bei einem solchen Projekt zwangläufig ergeben.
Wir waren uns zu beginn noch nicht bewusst, welche Funktionalitäten wichtig sind und welche weniger.
Der ursprüngliche Plan so schnell wie möglich einen Roundtrip durch alle Systeme zu erlangen.
Als wir gemerkt haben, dass wir es mit vielen verschiedenen Schnittstellen zu tun haben, sind wir von diesem Plan wieder abgekommen und haben uns dazu entschlossen, zuerst unsere App zu stabilisieren und dann eine Schnittstelle nach der anderen anzuschauen.

Mit Hilfe der \gls{ci} hatten wir stets ein lauffähiges System welchen automatisiert gebaut und ausgerollt wurde.