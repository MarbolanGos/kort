\subsection{Sprint 1}

\textbf{01. Oktober 2012 bis 14. Oktober 2012}

Alle Informationen zum Sprint 1 sind auch in unserem Wiki zu finden:
\url{http://kort.rdmr.ch/redmine/projects/kort/wiki/Sprint_1}

\subsubsection{Hauptaufgaben / Fokussierung im Sprint}

\begin{itemize}
	\item Aufsetzen der Infrastruktur (Repository, Projektmanagement, Entwicklungsumgebung, Server)
	\item Einarbeitung in die Thematik
	\begin{itemize}
		\item Gamification
		\item Highscore API (?)
		\item OpenStreetMap
		\item OpenLayers
		\item allenfalls weitere \gls{API}s
	\end{itemize}
	\item Wrapper für OpenStreetMap in Sencha Touch
	\item Projektsetup
	\begin{itemize}
		\item GitHub\footnote{\url{http://www.github.com}} als Repository
		\item \LaTeX{}\footnote{\url{http://www.latex-project.org}} für Dokumentation
		\item Heroku\footnote{\url{http://www.heroku.com}} fürs Hosting
		\item Travis \footnote{\url{http://travis-ci.org}} für Continuous Integration (CI)
		\item Redmine \footnote{\url{http://www.redmine.org}} für Projektmanagement/Wiki/Bugtracker
		\item Scrum \footnote{\url{http://de.wikipedia.org/wiki/Scrum}} als Methodik
	\end{itemize}
\end{itemize}

\subsubsection{Ziele}
\begin{itemize}
	\item OpenStreetMap genauer kennenlernen (Daten, Struktur, API) können
	\item Integration in Sencha sicherstellen
	\item Konzept der \emph{Gamification} erarbeiten
\end{itemize}

\subsubsection{Abgabe / Deliverables}

\begin{itemize}
	\item Übersicht zum Projekt hier auf Redmine (Sprints, Issues/Tasks, generelle Infos)
	\item Projekt-Setup mit GitHub, Heroku und Jenkins
	\item Lauffähiger Prototyp mit OpenStreetMap in Sencha Touch
\end{itemize}

\subsubsection{Erledigte Arbeiten}
Im ersten Sprint konnten wir bereits einiges erledigen. So gelang uns ein automatisierter Build der App zu Heroku. Ausserdem haben wir eine Sencha Touch Komponente für OpenLayers erstellt, um OpenStreetMap Daten in der App darstellen zu können. Ebenfalls gelang es uns mit dem Plugin \emph{Ext.i18n.bundle-touch} die Internationalisierung der App vorzubereiten.

\subsubsection{Probleme}
Leider mussten wir feststellen, dass OpenLayers Library eine nicht mehr zeitgemässes \gls{API} aufweist. Wir werden deshalb im zweiten Sprint nach Alternativen dafür suchen.