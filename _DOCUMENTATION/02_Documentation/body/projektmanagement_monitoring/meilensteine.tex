\section{Meilensteine}
\label{meilensteine}

Nachdem wir die Hauptaktivitäten des Projekts abschätzen konnten, haben wir eine Liste von Meilensteinen definiert.
Diese beschreiben konkrete Ziele welche während dieser Arbeit zu erreichen sind.

Wir haben uns jeweils einige Meilensteine für einen Sprint zum Ziel gesetzt und haben über die Wiki-Seite unseren Fortschritt festgehalten.

Somit gab es neben dem rein agilen Sprint-Planning auch noch funktionale Eckpfeiler.

\subsection{Checkliste}
\subsubsection{Meilenstein 1: Aufbereiten von Fehlerdaten}
\tick Geeignete Fehler finden \\
\tick Fehler in Datenbank speichern \\
\tick Fehler filtern \\
\tick Fehler zugänglich machen

\subsubsection{Meilenstein 2: Lesen von Fehlerdaten}

\tick Zugriff auf Datenbank aus App \\
\tick Persistieren der Daten in App

\subsubsection{Meilenstein 3: Kartenanzeige von Fehlern}

\tick Geeignete Kartendarstellung von Fehler finden \\
\tick Fehler auf der Karte anzeigen \\
\tick Location-based Karte

\subsubsection{Meilenstein 4: OAuth}

\tick Login möglich über OAuth \\
\tick User wird in App persistiert \\
\tick Logout möglich \\
\cross Login mit OSM-OAuth

\subsubsection{Meilenstein 5: Schreiben von Fehlerbehebungen}

\tick Input von User speichern in DB \\
\tick Unterscheidung von Fehlertypen

\subsubsection{Meilenstein 6: Verifizieren von Änderungen}

\tick UI zum verifizieren von Fehlerbehebungen \\
\tick Implementation eines Thresholds für Verfikationen

\subsubsection{Meilenstein 7: Daten zu OSM schicken}

\cross Verifikationsdaten einheitlich via API an OSM senden \\
\cross Prüfung ob eine Änderung zulässig ist

\subsubsection{Meilenstein 8: Gamification-Konzepte (Highscore, Leaderboard, Badges, Achievements)}

\tick Erarbeitung eines Gamifications-Konzepts für OSM \\
\tick Geeignete Elemente in App umsetzen (Punktesystem, Highscore, Badges)

\subsection{Fazit}
Bereits die Meilensteinliste vermittelt ein ambitioniertes Ziel, welches zu erreichen war.
Leider konnten wir nicht alle Punkte abschliessen.
Schlussendlich haben wir uns auf die Grundfunktionalität der \gls{WebApp} konzentriert.
So mussten wir zum Beispiel das Hinzufügen eines weiteren \gls{OAuth}-Anbieters verzichten.

Sehr schade war, dass wir keine Zeit mehr hatten, um die korrigierten Daten zu \gls{OpenStreetMap} zurückzuschicken.