\subsection{Sprint 4}

\textbf{12. November 2012 bis 2. Dezember 2012}

Alle Informationen zum Sprint 4 sind auch in unserem Wiki zu finden:
\url{http://kort.rdmr.ch/redmine/projects/kort/wiki/Sprint_4}

\subsubsection{Hauptaufgaben / Fokussierung im Sprint}

\begin{itemize}
	\item Backend
	\begin{itemize}
		\item Datenbank-Setup
		\item OAuth-Handling auf Server (Refresh-Token)
	\end{itemize}
	\item Frontend
	\begin{itemize}
		\item Validations-Maske erstellen mit Fix-Einträgen, welche zu verifizieren sind
		\item Highscore-Masken
		\item Abschluss Bug Detailmasken
	\end{itemize}
\end{itemize}

\subsubsection{Ziele}
\begin{itemize}
	\item Validations-Maske implementiert
	\item Highscore-Maske implementiert
	\item Komplettes Datenbank-Setup
	\item User-Handling (Client, Server, Persistenz, Refresh-Token)
\end{itemize}

\subsubsection{Abgabe / Deliverables}

\begin{itemize}
	\item Lauffähiger Prototyp
	\begin{itemize}
		\item Validations-Maske zeigt zu validierenden Lösungsvorschläge an
		\item Highscore (global) kann angezeigt werden
		\item Datenbank mit vollständigem Schema (bootstrapped)
	\end{itemize}
\end{itemize}

\subsubsection{Erledigte Arbeiten}
Zu Beginn des Sprints mussten wir die Datenbank nach unserem definierten Schema aufbauen.
Dadurch konnten wir mit der Implementation der weiteren Masken der App (Validation, Highscore) starten.
Zusätzlich haben wir begonnen die Benutzeranmeldungen in der Datenbank zu persistieren.
Während dieses Sprints nahmen wir noch am OSM Stammtisch\footnote{\url{http://wiki.openstreetmap.org/wiki/DE:Switzerland:Z\%C3\%BCrich/OSM-Treffen\#36._OSM-Stammtisch}} teil und konnten dort unsere App der \glslink{Mapper}{Mapping}-Community präsentieren.
Wir erhielten spannende und wichtige Hinweise für die weitere Entwicklung unserer App.

\subsubsection{Probleme}
Leider konnten wir die Persistierung der Benutzer noch nicht komplett abschliessen.
Wir werden dies im nächsten Sprint angehen.