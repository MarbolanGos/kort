% Neue Seite beginnen
\cleardoublepage

% Stil des Abstract-Titels veraendern
\renewcommand{\abstractname}{{\Huge\bfseries Abstract}}
% Titel auch in Kopfzeile anzeigen
\markboth{Abstract}{Abstract}

\begin{abstract}
% Kopf- und Fusszeile auch auf Abstractseite
\thispagestyle{scrheadings}

Das Ziel dieser Arbeit war das Erstellen einer cross-platform \gls{WebApp} zur Behebung von Fehlern in den OpenStreetMap-Daten.

Mit der App lassen sich verschiedene Typen von Fehlern beheben.
Dies kann zum Beispiel ein fehlendes Geschwindigkeitslimit bei einer Strasse oder ein Point of Interest ohne Namen sein.

Die gemachten Lösungsvorschläge der Benutzer werden dann von anderen Benutzern validiert.
Damit lässt sich die Qualität der Daten sicherstellen.
Wurde ein Lösungsvorschläg genügend oft positiven bewertet, so gilt dieser als akzeptiert. 
Solche Korrekturen werden dann zurück zu OpenStreetMap geschickt und somit die Karte für alle verbessert.

Um einen Anreiz zu bieten, die App längerfristig zu verwenden, wurden verschiedene Game-Elemente eingebaut.
So lassen sich beispielsweise mit dem Lösen von Fehlern Punkte sammeln und Auszeichnungen gewinnen.
Zusätzlich kann man sich über eine Highscore mit den anderen Spielern vergleichen.

Dieses Konzept, welches auch als Gamification bezeichnet wird, wurde im Bericht noch weiter beschrieben.
Es wurden weitere mögliche Elemente, welche in diesem Zusammenhang sinnvoll wären, untersucht und dokumentiert.

\end{abstract}
