% Neue Seite beginnen
\cleardoublepage

% Stil des Abstract-Titels veraendern
\renewcommand{\abstractname}{{\Huge\bfseries Abstract}}
% Titel auch in Kopfzeile anzeigen
\markboth{Abstract}{Abstract}

\begin{abstract}
% Kopf- und Fusszeile auch auf Abstractseite
\thispagestyle{scrheadings}

Das Ziel dieser Arbeit war das Erstellen einer cross-platform WebApp zur Behebung von Fehlern in den OpenStreetMap-Daten.

Mit der App lassen sich verschiedene Typen von Fehlern beheben.
Dies kann zum Beispiel ein fehlendes Geschwindigkeitslimit bei einer Strasse oder ein Point of Interest ohne Bezeichnung sein.
Die Lösungsvorschläge müssen dann von weiteren Benutzern validiert werden.
Damit lässt sich die Qualität der Daten sicherstellen.
Ist ein Minimum an positiven Bewertungen vorhanden, werden die eingetragenen Daten zu OpenStreetMap zurückgeschrieben.

Um einen Anreiz zu bieten, die App längerfristig zu verwenden, wurden verschiedene Game-Elemente eingebaut.
So lassen sich beispielsweise mit dem Lösen von Fehlern Punkte sammeln und Auszeichnungen gewinnen.
Zusätzlich kann man sich über eine Highscore mit den anderen Spielern vergleichen.

Dieses Konzept, welches auch als Gamification bezeichnet wird, wurde im Bericht noch weiter beschrieben.
Es wurden weitere mögliche Elemente, welche für diesen Anwendungsfall sinnvoll wären untersucht und dokumentiert.

\end{abstract}
