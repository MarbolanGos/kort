% Typ des Dokuments
\documentclass[abstracton, a4paper, 12pt]{scrreprt}

% Encoding (utf8)
\usepackage[utf8]{inputenc}
\usepackage[T1]{fontenc}

% Silbentrennung (Neu-Deutsch)
\usepackage[ngerman]{babel}

% Literaturverzeichnis (Deutsch)
\usepackage{bibgerm}
% Spezialseiten (Literarturverzeichnis, Abbildungsverzeichnis, usw.) in Inhaltsverzeichnis anzeigen
\usepackage[nottoc]{tocbibind}

% Farben
\usepackage{color}
\usepackage[table]{xcolor}
\definecolor{darkgreen}{rgb}{0,0.6,0}
\definecolor{darkgrey}{rgb}{0.5,0.5,0.5}
\definecolor{grey}{rgb}{0.8,0.8,0.8}
\definecolor{lightgrey}{rgb}{0.95,0.95,0.95}
\definecolor{mauve}{rgb}{0.58,0,0.82}

% Schriften
\usepackage{pifont}
\newcommand{\tick}{\ding{51}\hspace{0.2cm}}
\newcommand{\cross}{\ding{55}\hspace{0.2cm}}

% Grafiken
\usepackage[pdftex]{graphicx}
\usepackage{epsfig}
% Umfliessen von Text um Tabellen und Bilder
\usepackage{wrapfig}

% Grafiken korrekt positionieren
\usepackage{float}
\restylefloat{figure}
\usepackage[section]{placeins}
\usepackage{subfigure}
% Zahlen verwenden für Subfigure counter
\renewcommand{\thesubfigure}{(\arabic{subfigure})}

% Hyperlinks und URLs
\usepackage[hyphens]{url}
\usepackage{hyperref}
\hypersetup{
   colorlinks,%
   citecolor=blue,%
   filecolor=blue,%
   linkcolor=blue,%
   urlcolor=blue
}
\urlstyle{same}

% Absatz
\setlength{\parindent}{0pt} % Absatzeinzug
\setlength{\parskip}{10pt} % Absatzabstand

% Glossar
\usepackage[toc]{glossaries}
\makeglossaries

% Kontrolle über Listen-Eigenschaften
\usepackage{enumitem}

% Abstaende bei Ueberschriften
\usepackage{titlesec}
% \titlespacing*{command}{left}{before-sep}{after-sep}[right-sep]
\titlespacing{\section}{0em}{12pt}{3pt}
\titlespacing{\subsection}{0em}{10pt}{2pt}
\titlespacing{\subsubsection}{0em}{8pt}{0em}

% TODO Kommentare
\usepackage{todonotes}

%%%%%%%%%%%%%%%%%%%%%%%%%%%%%%%%%%%%%%%%%%%%%%%%%%%
% Kopf- und Fusszeile
%%%%%%%%%%%%%%%%%%%%%%%%%%%%%%%%%%%%%%%%%%%%%%%%%%%
% Seitenränder
\usepackage[inner=2.2cm,outer=2.2cm,top=1.7cm,bottom=1.7cm,includeheadfoot]{geometry}

% Kopf- und Fusszeile mit Linien
\usepackage[automark,headsepline,footsepline]{scrpage2}

\pagestyle{scrheadings}
% Kopf- und Fusszeile auch bei Kapitel- und Partsanfangsseiten
\renewcommand*{\chapterpagestyle}{scrheadings}
\renewcommand*{\partpagestyle}{scrheadings} 

% Kopf- und Fusszeile leeren
\clearscrheadfoot

% Inhalt der Kopfzeile
\ihead{\footnotesize{\leftmark}}

% Inhalt der Fusszeile
\ifoot{\footnotesize{Gamified Mobile App für die Verbesserung von OpenStreetMap}}
\ofoot{\footnotesize{\thepage}}

% MakeUppercase überschreiben, um Grosschreibung in Kopfzeile für Spezialseiten zu deaktivieren (Achtung böser Hack!)
\renewcommand*\MakeUppercase[1]{#1}

%%%%%%%%%%%%%%%%%%%%%%%%%%%%%%%%%%%%%%%%%%%%%%%%%%%
% Tabellen
%%%%%%%%%%%%%%%%%%%%%%%%%%%%%%%%%%%%%%%%%%%%%%%%%%%
% Für Tabellen, welche über mehrere Seiten gehen
\usepackage{longtable}

% Mehrere Spalten zusammenfassen
\usepackage{hhline}

% Padding links und rechts von Zelle
\setlength{\tabcolsep}{5px}
% Padding oben und unten (mittels arraystretch)
\renewcommand{\arraystretch}{1.3}

% Variabeln für Tabellenbreiten definieren
% 2-spaltige Tabelle
\newlength{\twocelltabwidth}
\setlength{\twocelltabwidth}{\textwidth}
\addtolength{\twocelltabwidth}{-4\tabcolsep - 3px} % subtrahiere 4x Padding (\tabcolsep) und 3 Rahmen

% 3-spaltige Tabelle
\newlength{\threecelltabwidth}
\setlength{\threecelltabwidth}{\textwidth}
\addtolength{\threecelltabwidth}{-6\tabcolsep - 4px} % subtrahiere Padding (\tabcolsep) und Rahmen

% 4-spaltige Tabelle
\newlength{\fourcelltabwidth}
\setlength{\fourcelltabwidth}{\textwidth}
\addtolength{\fourcelltabwidth}{-8\tabcolsep - 5px} % subtrahiere Padding (\tabcolsep) und Rahmen

%%%%%%%%%%%%%%%%%%%%%%%%%%%%%%%%%%%%%%%%%%%%%%%%%%%
% Syntaxhighlighter
%%%%%%%%%%%%%%%%%%%%%%%%%%%%%%%%%%%%%%%%%%%%%%%%%%%
% Syntaxhighlighter (benoetigt color und xcolor package)
\usepackage{listings}
\renewcommand{\lstlistingname}{Code-Ausschnitt}

\lstset{ %
  language=HTML,                % the language of the code
  basicstyle=\footnotesize,           % the size of the fonts that are used for the code
  numbers=left,                   % where to put the line-numbers
  numberstyle=\tiny\color{darkgrey},  % the style that is used for the line-numbers
  stepnumber=1,                   % the step between two line-numbers. If it's 1, each line will be numbered
  numbersep=5pt,                  % how far the line-numbers are from the code
  backgroundcolor=\color{white},  % choose the background color. You must add \usepackage{color}
  showspaces=false,               % show spaces adding particular underscores
  showstringspaces=false,         % underline spaces within strings
  showtabs=false,                 % show tabs within strings adding particular underscores
  frame=single,                   % adds a frame around the code
  rulecolor=\color{darkgrey},        % if not set, the frame-color may be changed on line-breaks within not-black text (e.g. commens (green here))
  tabsize=2,                      % sets default tabsize to 2 spaces
  captionpos=b,                   % sets the caption-position to bottom
  breaklines=true,                % sets automatic line breaking
  breakatwhitespace=false,        % sets if automatic breaks should only happen at whitespace
  title=\lstname,                   % show the filename of files included with \lstinputlisting;
                                  % also try caption instead of title
  keywordstyle=\color{blue},          % keyword style
  commentstyle=\color{darkgreen},       % comment style
  stringstyle=\color{mauve},         % string literal style
  escapeinside={\%*}{*)},            % if you want to add a comment within your code
  morekeywords={*,...}               % if you want to add more keywords to the set
}

% Javascript Syntaxhighliting
\lstdefinelanguage{JavaScript} {
	morekeywords={
		break,const,continue,delete,do,while,export,for,in,function,
		if,else,import,in,instanceOf,label,let,new,return,switch,this,
		throw,try,catch,typeof,var,void,with,yield
	},
	sensitive=false,
	morecomment=[l]{//},
	morecomment=[s]{/*}{*/},
	morestring=[b]",
	morestring=[d]'
}
\lstset{
	frame=tb,
	framesep=5pt,
	basicstyle=\footnotesize\ttfamily,
	showstringspaces=false,
	keywordstyle=\ttfamily\bfseries\color{blue},
	identifierstyle=\ttfamily,
	stringstyle=\ttfamily\color{mauve},
	commentstyle=\color{darkgreen},
	rulecolor=\color{darkgrey},
	xleftmargin=5pt,
	xrightmargin=5pt,
	aboveskip=\bigskipamount,
	belowskip=\bigskipamount
}

\lstdefinestyle{examples}
{numbers=none, frame=none}

% Inline Code-Formatierung
\newcommand{\inlinecode}{\texttt}

% Silbentrennung
\hyphenation{Web-app-li-ka-tion}